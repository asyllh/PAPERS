\documentclass{elsarticle}
%---- PREAMBLE ----%

% Allows separate files to be used in the main file
\usepackage{subfiles}

% Algorithms
\usepackage{algorithm}
\usepackage[noend]{algpseudocode}
\renewcommand{\algorithmicrequire}{\textbf{Input:}}
\renewcommand{\algorithmicensure}{\textbf{Output:}}
\newcommand{\algorithmicbreak}{\State \textbf{break}}
\newcommand{\Break}{\algorithmicbreak}
\renewcommand{\algorithmicreturn}{\State \textbf{return}}
\newcommand{\algorithmicto}{\textbf{ to }}
\newcommand{\To}{\algorithmicto}
\newcommand{\algorithmicrun}{\State \textbf{call }}
\newcommand{\Run}{\algorithmicrun}
\newcommand{\algorithmicoutput}{\textbf{output }}
\newcommand{\Output}{\algorithmicoutput}
\newcommand{\algorithmictrue}{\textbf{true}}
\newcommand{\True}{\algorithmictrue}
\newcommand{\algorithmicfalse}{\textbf{false}}
\newcommand{\False}{\algorithmicfalse}

% Contains advanced math extensions
\usepackage{amsmath}

% Introduces the *proof* environment and the \theoremstyle command
\usepackage{amsthm}

% Adds new symbols to be used in math mode, e.g. \mathbb
\usepackage{amssymb}

% To declare multiple authors
%\usepackage{authblk}

% Provides extra comands as well as optimisation for producing tables
\usepackage{booktabs}
\newcommand{\ra}[1]{\renewcommand{\arraystretch}{#1}}

% Allows customisation of appearance and placements for figures/tables etc.
\usepackage{caption}

% Adds support for arbitrarily-deep nested lists
\usepackage[inline]{enumitem}

%Support for changing size of font in table footnotes
\usepackage{etoolbox}

% Improves the interface for defining floating objects such as figures/tables
\usepackage{float}

\usepackage{fullpage}

% For easy management of document margins and the document page size
%\usepackage[a4paper]{geometry}

% Allows insertion of graphic files within a document
\usepackage{graphicx}

% Manage links within the document or to any URL when you compile in PDF
\usepackage[colorlinks]{hyperref} 
\usepackage[dvipsnames]{xcolor}
%Tikz colours, used in tikz figures only
\colorlet{tBlue}{RoyalBlue!35!Cerulean} %tikz color
\colorlet{tRed}{Red} %tikz color
\definecolor{tGreen}{HTML}{569909} %tikz color
\definecolor{tOrange}{HTML}{FA7602} %tikz color

%Text colours
\colorlet{myRed}{Red!50!OrangeRed}
\definecolor{myOrange}{HTML}{FA7602}
\definecolor{myGreen}{HTML}{569909}
\definecolor{myAqua}{HTML}{00B1BA} %02BEB8
\definecolor{myBlue}{HTML}{0095FF} %00B3FF
\colorlet{myPurple}{Orchid} 
\colorlet{myPink}{Rhodamine!65!Lavender}
\colorlet{myGray}{Gray!90!White}

% The document class `elsarticle` uses the \AtBeginDocument comman to define the colour of all link types as blue, so we use the \AtBeginDocument command again to overwrite the colour settings to the colors we choose. REMEMBER TO REMOVE THIS BEFORE SUBMITTING.
\AtBeginDocument{% 
	\hypersetup{
		linkcolor=myPurple,
		citecolor=myAqua,
		urlcolor=black
	}
} %end of \AtBeginDocument

\newcommand{\intro}[1]{{\color{myOrange}#1}} % \intro{<text>}, makes <text> orange, use to highlight things to do, urgent, or revisit in introduction section

\newcommand{\ahc}[1]{{\color{myBlue}#1}} % \ahc{<text>}, makes <text> blue, use to highlight things to do, urgent, or revisit in AHC section

\newcommand{\heur}[1]{{\color{myPink}#1}} % \scspp{<text>}, makes <text> pink, use to highlight things to do, urgent, or revisit in SCSPP section

\newcommand{\ea}[1]{{\color{myRed}#1}} % \ea{<text>}, makes <text> red, use to highlight things to do, urgent, or revisit in EA section

\newcommand{\cmsa}[1]{{\color{myGreen}#1}} % \cmsa{<text>}, makes <text> green, use to highlight things to do, urgent, or revisit in CMSA section

\newcommand{\conc}[1]{{\color{myPurple}#1}} % \conc{<text>}, makes <text> purple, use to highlight things to do, urgent, or revisit in Conclusion section

\newcommand{\note}[1]{{\color{myPurple}#1}} % \note{<text>}, makes <text> turquoise, use to highlight things to do, urgent, or revisit in any section


\newcommand{\alert}[1]{{\color{myRed}#1}} % \alert{<text>}, makes <text> red, use to highlight things to do, urgent, or revisit.

\newcommand{\done}[1]{{\color{myGray}#1}} % \done{<text>}, makes <text> gray, use to highlight things that are not required or should be ignored.

\newcommand{\ialert}[1]{{\color{myRed}\item#1}} % \ialert{<text>}, makes <text> a red bullet point, use for bullet points of things to do, urgent, or revisit.

\newcommand{\idone}[1]{{\color{myGray}\item#1}} % \idone{<text>}, makes <text> a gray bullet point, use for bullet points that have been completed, that are not required or should be ignored.

\renewcommand{\idone}[1]{} % hides all \idone bullet points

% Successor of amsmath
\usepackage{mathtools}

\usepackage{multirow}

% No indentation, space between paragraphs
%\usepackage{parskip}

\usepackage{natbib}

%Include standalone .tex files
\usepackage{standalone}

% Define multiple floats (figures/tables) within one environment with individual captions 1a, 1b etc
\usepackage{subcaption}

\usepackage{tabularx}

\usepackage{threeparttable}
\appto\TPTnoteSettings{\footnotesize} % Change font size of table footnotes to footnotesize

\usepackage{tikz}

\usepackage{wrapfig}

\DeclarePairedDelimiter{\floor}{\lfloor}{\rfloor}
\DeclarePairedDelimiter{\ceil}{\lceil}{\rceil}

%Theorem style
\theoremstyle{plain}% default
\newtheorem{theorem}{Theorem}
%\newtheorem{corollary}{Corollary}
\newtheorem{definition}{Definition}

%\theoremstyle{definition}
%\newtheorem{definition}{Definition}
%\newtheorem{proposition}{Proposition}
%\newtheorem{exmp}{Example}[section]
\begin{document}
	
\begin{frontmatter}
\title{An Evolutionary Algorithm for the Score-Constrained Strip-Packing Problem}
\author{Asyl L. Hawa}
\author{Rhyd Lewis}
\author{Jonathan M. Thompson}
\address{School of Mathematics, Cardiff University, Senghennydd Road, Cardiff, UK}
\begin{abstract}
\note{Type of packing problem in which order and orientation of items is important. Paper investigates heuristics, EA, postoptimisation.} 
\end{abstract}	
\end{frontmatter}

%--------------------------------------------------------------------------------------
\section{Introduction}
\label{sec:intro}
\noindent In operational research, there are many problems that require the grouping of elements into subsets. These types of problems can be seen in areas such as scheduling \cite{thompson1998, carter1996}, frequency assignment \cite{aardal2007}, graph colouring \cite{lewis2012, malaguti2008}, and load balancing \cite{rekiek1999}. Formally, given a set $\mathcal{I}$ of $n$ elements, a grouping problem involves producing a partition of subsets $\mathcal{S} = \{S_1, S_2,...,S_k\}$ such that
\begin{subequations}
	\begin{alignat}{2}
	\bigcup\nolimits_{i=1}^{k} S_i &= \mathcal{I}, & \label{eqn:allpacked}\\[3pt]
	S_i \cap S_j &= \emptyset &\quad &\forall \hspace{1mm} i, j \in \{1, 2,\dotsc,k\}, \hspace{1mm} i \neq j, \label{eqn:nointersect}\\[3pt]
	S_i &\in \mathcal{F} & &\forall \hspace{1mm} i \in \{1,2,\dotsc,k\}.\label{eqn:feasible}
	\end{alignat}
\end{subequations}

\noindent Equations \eqref{eqn:allpacked} and \eqref{eqn:nointersect} state the requirement that every element $i \in \mathcal{I}$ must be in exactly one of the $k$ subsets, while \eqref{eqn:feasible} specifies that each subset $S_i \in \mathcal{S}$ must be feasible. The set $\mathcal{F}$ is used to denote the set of all feasible subsets of elements in $\mathcal{I}$. The notion of feasibility is dependent on the specific constraint of the given problem.

Some problems require a solution to be partitioned into exactly $k$ subsets, whilst others have the task of minimising the number of subsets $k$ required to partition the elements provided conditions \eqref{eqn:allpacked}--\eqref{eqn:feasible} are satisfied. The latter problems are known as Minimum Grouping Problems (MGPs), named so by Lewis \cite{lewis2009}. One particular type of MGPs are packing problems. The classical one-dimensional bin-packing problem (BPP) requires a set of items of varying sizes to be packed into the fewest number of bins of fixed capacity. It can be seen that for this problem $\mathcal{F}$ contains all subsets of items whose total size does not cause the bin to be overfilled. Although seemingly simple by definition, the BPP is NP-hard and has been widely researched \cite{coffman1984, decarvalho1999, fleszar2002}. From the BPP stems numerous packing problems with various adaptations \cite{coffman1987, haouari2009}.

One interesting problem related to the BPP is the Trapezoid Packing Problem (TPP), initially investigated by Lewis et al. \cite{lewis2011}. In the roofing industry, trusses are to be cut from wooden boards so as to minimise the number of boards required. These trusses, however, are trapezoidal in shape, which introduces the additional challenge of minimising the amount of triangular waste material between adjacent trapezoids on each board by rotating and reordering the items. In 2017, Lewis and Holborn \cite{lewis2017} presented the Euler-Splice algorithm which, given a subset of trapezoidal items whose total size is less than or equal to the board size, is able to determine whether a feasible ordering of the items on the board exists in polynomial time.

Another problem similar to the BPP is the cutting stock problem (CSP). One particular case described by Garraffa et al. \cite{garraffa2016} considers sequence-dependent cut-losses (SDCL). Rectangular items of varying lengths are to be cut from strips of material of fixed lengths, however the type of cutting machine used results in material loss between items during the cutting process. The amount of these ``cut losses'' varies between different items, and is also dependent on the order of the items: a cut loss between two adjacent items A and B, with A packed first, may not necessarily be equal to the cut loss that arises when B is packed before A. Thus, the CSP-SDCL involves packing the items onto the fewest number of strips such that the sum of item lengths \emph{and} the sum of cut losses between all adjacent items on each strip does not exceed the strip capacity.

The focus of this paper is on a further problem involving order and orientation that occurs in the packaging industry where flat pieces of cardboard are to be folded into boxes. This strip-packing problem was introduced as an open-combinatorial problem by Goulimis in 2004 \cite{goulimis2004}. A set $\mathcal{I}$ of $n$ rectangular cardboard items of equal height $H$ and varying widths are to be packed onto a strip $S$ of height $H$ from left to right. Each item $i \in \mathcal{I}$ is marked with two vertical score lines in predetermined places. The distance between each score line and the edge of the item are the score widths, $a_i, b_i \in \mathbb{Z}^+$ (where $a_i \leq b_i$). Once the items are packed onto the strip, a pair of knives mounted on a bar simultaneously cut along the score lines of two adjacent items, making it easier to fold the cardboard at a later stage. Due to the manner in which the knives are secured onto the bar, the knives must maintain a set distance from one another, a so-called ``minimum scoring distance'' $\tau \in \mathbb{Z}^+$, which is approximately 70mm in industry. Therefore, in order for the knives to score all of the items in the correct locations, the distance between two score lines of adjacent items must exceed the minimum scoring distance. Hence, the following \emph{vicinal sum constraint} must be fulfilled:
\begin{equation}
	\label{eqn:vsc}
	\textbf{rhs}(i) + \textbf{lhs}(i+1) \geq \tau \quad \forall \hspace{1mm} i \in \{1,2,\dotsc,|S|- 1\},
\end{equation}

\noindent where \textbf{lhs}($i$) and \textbf{rhs}($i$) denote the left- and right-hand score widths of the $i$th item on the strip. Clearly if this constraint is met the distance between the score lines will be sufficient for the knives to be able to cut appropriately.

Each item $i \in \mathcal{I}$ can be packed onto the strip in a regular orientation, denoted $(a_i, b_i)$, where the smaller score width $a_i$ is on the left-hand side of item $i$, or in a rotated orientation $(b_i, a_i)$, where the item has been rotated $180^{\circ}$ so that the larger score width $b_i$ is on the left-hand side. Thus, the problem involves finding an ordering and orientation of the items on the strip such that the sum of all adjacent score widths is greater than or equal to the minimum scoring distance $\tau$, i.e. the vicinal sum constraint is satisfied.\footnote{Note that the outermost score widths on a strip $S$ are disregarded as they are not adjacent to any other items.} As there are $2^{n-1} n!$ distinct orderings of $n$ items, an increase in the problem size can easily lead to a combinatorial explosion. Figure~\ref{fig:itemsknife} shows how adjacent items on a strip are scored simultaneously by pairs of knives. Although the vicinal sum constraint is met between items A and B, the full alignment of al three items on the strip is infeasible as the sum of the adjacent score widths of items B and C is less than the minimum scoring distance $\tau$, and so the knives are unable to move close enough together to score the lines in the required locations. Observe that a feasible alignment of all three items on the strip can be obtained by rotating item C.

\begin{figure}[H]	
	\centering
	\includestandalone[width=0.75\textwidth]{figures/itemsknife}
	\caption{Dimensions of an item $i$ marked with dashed score lines, and an example packing showing both feasible and infeasible alignments of three items to be scored by pairs of knives. Here, the minimum scoring distance $\tau = 7$.}	
	\label{fig:itemsknife}
\end{figure}

\noindent Notice that for this single-strip problem the items' widths are ignored as the capacity of the strip $S$ is not provided; thus it is assumed that $S$ is able to accommodate all items in $\mathcal{I}$. Suppose now, \intro{as is usual in industry}, that the strips have a fixed finite width. \intro{A more generalised problem can then be formulated as follows}:

\begin{definition} % SCSPP
	\label{defn:scspp}
	Let $\mathcal{I}$ be a set of $n$ rectangular items of height $H$ with varying widths $w_i \in \mathbb{Z}^+$ and score widths $a_i, b_i \in \mathbb{Z}^+$ for each item $i \in \mathcal{I}$. Given a minimum scoring distance $\tau \in \mathbb{Z}^+$, the Score-Constrained Strip-Packing Problem (SCSPP) involves packing all items in $\mathcal{I}$ onto the fewest number of strips of height $H$ and width $W$ such that the vicinal sum constraint is satisfied on each strip and no strip is overfilled.
\end{definition}

\noindent The BPP can be seen as a special case of the SCSPP when $\tau = 0$. As with the TPP and CSP-SDCL, the SCSPP not only involves deciding which strip each item should be packed onto, but also, unlike the BPP, \emph{how} the items should be packed -- that is, determining the order and orientation of items within each strip. One specific difference, however, concerns the feasibility of individual strips. In the TPP, although clearly not optimal, it is still legal to place trapezoids with opposite angles, i.e. `$\backslash$' and `/', alongside one another. Likewise in the CSP-SDCL, two items with a large cut loss between them can still be packed alongside one another if necessary. Both of these problems allow items to be packed in \emph{any} order and orientation as long as the strips are not overfilled. In contrast, the SCSPP possesses the strong vicinal sum constraint which if violated immediately causes an alignment of items on a strip to be invalid, thus rendering the entire solution infeasible. For consistency, we shall refer to the \intro{single-strip} problem of finding a legal sequence of a subset of items $\mathcal{I}' \subseteq \mathcal{I}$ on a \intro{single strip} as the Score-Constrained Strip-Packing Sub-Problem (SubSCP).

In the next section, we will provide a brief overview of the polynomial-time algorithm used to solve the SubSCP. Section~\ref{sec:scspp} will explain the difficulties associated with the SCSPP and analyse bespoke heuristics \intro{created in a different paper}. An evolutionary algorithm for the SCSPP is presented in Section~\ref{sec:ea}, along with results from rigorous experiments. Section~\ref{sec:postopt} details a postoptimisation procedure involving a recursive backtracking algorithm to improve upon results obtained from the EA, and finally Section~\ref{sec:conclusion} concludes the paper and discusses outcome and possible directions for further work.

{\color{myOrange}
\begin{itemize}[leftmargin=*]
	\item Change first sentence.
	\item For the TPP and SDCL, order and orientation helps minimise waste material and find a better solution using fewer strips, but for the SCSPP order and orientation is required to find a \emph{feasible} solution.
	\item Two phsae approach using postoptimisation stage (like Malaguti \cite{malaguti2008}).
	\item Items can be run through the machine individually but this incurs extra time/costs.
\end{itemize}
}

%--------------------------------------------------------------------------------------
\section{The Alternating Hamltonian Construction Algorithm}
\label{sec:ahc}
\noindent Firstly, let us consider the following sequencing problem originally defined by Hawa et al. \cite{hawa2018}:

\begin{definition} % COP
	\label{defn:cop}
	Let $\mathcal{M}$ be a multiset of unordered pairs of integers $\mathcal{M} = \{\{a_1, b_1\}, \{a_2, b_2\},\dotsc,\{a_n, b_n\}\}$, and let $\mathcal{T}$ be a sequence of the elements of $\mathcal{M}$ in which each pair is a tuple. Given a fixed value $\tau \in \mathbb{Z}^+$, the Constrained Ordering Problem (COP) consists of finding a solution $\mathcal{T}$ such that the sum of adjacent values from different tuples is greater than or equal to $\tau$.
\end{definition}

\noindent For example, given the COP instance $\mathcal{M} = \{\{1,2\}, \{1,6\}, \{2,3\}, \{2,4\}, \{3,5\}, \{4,5\}\}$ and a fixed value $\tau = 7$, one possible feasible sequence is $\mathcal{T} = \langle(1,6), (2,4), (3,5), (2,3), (4,5), (2,1)\rangle$. It is evident that the COP is in fact equivalent to the SubSCP, whereby each pair in $\mathcal{M}$ can be seen as an item $i$ represented by its score widths $a_i, b_i$, and the constraint value $\tau$ is the minimum scoring distance. It follows that the requirement for the sum of adjacent values to exceed $\tau$ corresponds to the vicinal sum constraint \eqref{eqn:vsc}.

We now present the Alternating Hamiltonian Construction (AHC) algorithm, a polynomial-time algorithm for solving the COP, and thus also the SubSCP. The underlying algorithm was originally proposed in \cite{becker2010}, and determines whether a feasible solution exists for a given instance. This was then extended by Hawa et al. \cite{hawa2018} so that if a solution does indeed exist AHC is able to formulate and present the final solution. This is particularly useful for problems in industry. Here, we further simplify and increase the efficiency and speed of AHC by replacing two of the subprocedures used in the previous version of AHC with a single algorithm.

We begin by producing a graphical representation of the given instance $\mathcal{M}$ of the COP. The graph $G$ has a vertex set $V$ defined using one vertex for each element in $\mathcal{M}$ in non-decreasing order, with each vertex $v_i \in V$ weighted with the value in $\mathcal{M}$ it represents. For simplicity, we shall refer to each value in $\mathcal{M}$ by its vertex on $G$. In order to prevent executing the algorithm unnecessarily, a basic preliminary test is performed to determine if the instance is infeasible. Of the $2n$ values, suppose the smallest values $v_1$ and $v_2$ are placed on the ends of the sequence. Clearly, if the next-smallest value $v_3$ and the largest value $v_{2n}$ do not meet the vicinal sum constraint, then there cannot exist a feasible ordering of all elements in $\mathcal{M}$. Note, however, that a positive outcome from this test does not necessarily imply that a feasible solution exists for the instance.

If $\mathcal{M}$ has not been deemed infeasible an extra pair of vertices is added to $G$, each assigned a value equal to $\tau$. $G$ comprises two edge sets: $B$, which contains \ahc{blue} edges between vertices that are \emph{partners}, that is, whose values make up an unordered pair in $\mathcal{M}$; and $R$, containing \ahc{red} edges between vertices that add up to $\geq \tau$ and are not partners. The additional vertices $v_{2n+1}$ and $v_{2n+2}$ are partners, and can be seen to be universal vertices as they are adjacent to every other vertex via an edge in $R$; thus both have degree $2n+1$. Given the bijective function $p : V \to V$ that associates each vertex $v_i \in V$ with its partner $p(v_i)$, the set of edges between partners can be denoted as $B = \{(v_i, p(v_i)) : v_i \in V\}$. Note that $|B| = n+1$, and so $B$ is a perfect matching. Figure~\ref{fig:threshold} illustrates the resulting graph $G = (V, B \cup R)$ produced from the example instance $\mathcal{M}$ of the COP provided above. The graph has a noticeable pattern, with the \ahc{degree of each vertex increasing in accordance with the weight} of the vertices - \ahc{degree sequence}.

A Hamiltonian cycle in a graph $G$ is a cycle that visits every vertex of $G$ exactly once. A graph containing such a cycle is said to be Hamiltonian. From this, we define a specific type of Hamiltonian cycle:

\begin{definition} % Alternating Hamiltonian Cycle
	\label{defn:althamcycle}
	Let $G = (V, B \cup R)$ be a simple, undirected graph where each edge is a member of one of two sets, $B$ or $R$. $G$ contains an alternating Hamiltonian cycle if there exists a Hamiltonian cycle such that successive edges alternate between sets $B$ and $R$.
\end{definition}

\noindent It can be seen that if two vertices are connected via an edge in $R$, then their corresponding values in $\mathcal{M}$ fulfil the vicinal sum constraint and can be ordered adjacent to one another, thus $R$ contains \emph{all} possible feasible pairings of values. It then follows that an alternating Hamiltonain cycle in $G$ corresponds to a legal sequence of the elements in $\mathcal{M}$, as the edges from $R$ depict the order of each tuple in $\mathcal{T}$ and the order of the values \emph{within} each tuple. The solution $\mathcal{T}$ must contain all elements from the problem instance $\mathcal{M}$ and the unordered pairs in $\mathcal{M}$ cannot be altered or rearranged, so all $n+1$ edges in $B$ must be present in the alternating Hamiltonian cycle. Therefore, the aim is to find a suitable set of edges $R' \subseteq R$ that, together with the edges in $B$, form an alternating Hamiltonian cycle in $G$. As $B$ is a perfect matching the subset $R'$ must also be a matching, else there will not be enough edges to form the cycle.

The universal vertices are used to aid the construction of the alternating Hamiltonian cycle as they are able to connect to the smallest vertices, however once an alternating Hamiltonian cycle has been produced the universal vertices and any incident edges are removed, leaving the smallest vertices as the endpoints on the path. These vertices correspond to the smallest values in the problem instance which will be in the outermost positions of the solution $\mathcal{T}$.

Determining whether a graph is Hamiltonian is one of the well-known 21 NP-complete problems \cite{karp1972}, whilst the problem of actually finding a Hamiltonian cycle is NP-hard. Consequently, the alternating Hamiltonian cycle problem is also NP-hard, as it is a generalisation of the Hamiltonian cycle problem \cite{haggkvist1977}. Despite this, due to the manner in which graphs are modelled from instances of the COP and the requirement that all edges in $B$ must be included, we are able to determine the existence of an alternating Hamiltonian cycle in a graph $G$ in polynomial time \cite{hawa2018}.

The first subprocedure of AHC is the Maximum Cardinality Matching (MCM) algorithm, which is used to produce a matching $R'$ from $R$ \cite{mahadev1994}. MCM takes each vertex $v_1, v_2,\dotsc,v_{2n+2}$ and adds to $R'$ the edge from $R$ connecting $v_i$ to the highest-indexed vertex $v_j$ that is not incident to an edge in already in $R'$. Pairs of vertices incident to edges in $R'$ are then said to be ``matched''. As with partners, the bijective function $m : V \to V$ associates with each vertex $v_i \in V$ its match, $m(v_i)$. This set can then be denoted as $R' = \{(v_i, m(v_i)): v_i \in V\}$. In the event that a vertex $v_i$ is not adjacent to any other vertex via an edge in $R$, the previous vertex $v_{i-1}$ can be rematched, provided 
\begin{enumerate*}[label={(\alph*)}]
	\item $i \neq 1$;
	\item $v_{i-1}$ has been matched; and
	\item $(v_{i-1}, p(v_i)) \in R$.
\end{enumerate*} 
Then, we simply set $m(v_i) = m(v_{i-1})$, and $m(v_{i-1}) = p(v_i)$.

If $R'$ does not contain $n+1$ edges, then no feasible solution exists. Otherwise, the subgraph $G'=(V, B \cup R')$ is a 2-regular graph consisting of cyclic components $C_1,C_2,\dotsc,C_z$, as show in Figures~\ref{fig:matching} and~\ref{fig:mps}. Clearly if $z = 1$ then $G'$ is an alternating Hamiltonian cycle, and a solution has been found. In the case of $z > 1$, these components must be connected together to form a single alternating Hamiltonian cycle.

\begin{figure}[H]	
	\centering
	\begin{subfigure}[h]{0.3\textwidth}
		\includestandalone[width=\textwidth]{figures/threshold}
		\caption{$G = (V, B \cup R)$}
		\label{fig:threshold}
	\end{subfigure} \hspace{5mm}
	\begin{subfigure}[h]{0.3\textwidth}
		\includestandalone[width=\textwidth]{figures/matching}
		\caption{$G' = (V, B \cup R')$}
		\label{fig:matching}
	\end{subfigure} \hspace{5mm}
	\begin{subfigure}[h]{0.243\textwidth}
		\includestandalone[width=\textwidth]{figures/mps}
		\caption{$G' = (V, B \cup R')$ planar}
		\label{fig:mps}
	\end{subfigure}
	\caption{Graph $G$ created from the example instance $\mathcal{M}$ provided in Section~\ref{sec:intro}, and the spanning subgraph $G'$ produced after MCM. In planar form, it can be seen that $G'$ comprises $z = 3$ cyclic components.}
	\label{fig:mcm}
\end{figure}

\noindent Recall that an edge in a graph is a \emph{bridge} if the removal of the edge increases the number of components of the graph. In order to connect the multiple cyclic components of $G'$, we must select edges from $R \backslash R'$ that are able to act as bridges between the components. The Bridge-Cover Recognition (BCR) algorithm produces sets of edges that can be used to combine the components of $G'$. A set $R_i$ is said to ``cover'' a component $C_j$ if $R_i$ contains an edge from $C_j$. BCR operates by creating sets of edges from $R'$ that cover all $z$ components of $G'$. 

In \cite{becker2010}, a procedure was used to find these sets of edges, and they determined whether a feasible solution existed for the given instance. However, for some instances, these sets were unable to connect the components of $G'$ into a single alternating Hamiltonian cycle, instead forming multiple cycles which do not correspond to a feasible solution. An additional procedure was implemented in \cite{hawa2018} that found and produced a feasible solution to these instances. Here, we simplify this method to improve speed and efficiency.

Firstly, the edges in $R'$ are sorted into a list $\mathcal{L}$ such that the lower-indexed vertices of the edges are in increasing order, and the higher-indexed vertices are in decreasing order. Any edges that cannot be sorted in this manner are removed from the list. Then, BCR searches through $\mathcal{L}$ for an edge whose lower-indexed vertex is adjacent to the higher-indexed vertex of the next edge via an edge from $R \backslash R'$, and that is not in the same component of $G'$ as the next edge. This edge starts the set $R_1$, and succeeding edges are added to $R_1$ provided the adjacency condition holds and the succeeding edge is not a member of a component that is already covered by $R_1$. In the event that $R_1 = \emptyset$ there are no suitable edges that can merge the components together, thus there is no feasible solution to the given instance $\mathcal{M}$. However, if $|R_1| = z$ then $R_1$ covers all $z$ components of $G'$, and a connecting procedure is able to use $R_1$ to combine the components into a single alternating Hamiltonian cycle.

Otherwise, $R_1$ is added to a collection $\mathcal{R}$, and the edges in $\mathcal{R}$ are removed from $\mathcal{L}$. An iterative procedure is then used to find overlapping sets. A set $R_i$ ``overlaps'' with $\mathcal{R}$ if $R_i$ covers exactly one component that is covered by $\mathcal{R}$, and all other components covered by $R_i$ are not yet covered by $\mathcal{R}$. During an iteration, BCR searches from the beginning of $\mathcal{L}$ for edges that fulfil the adjacency condition and can form an overlapping set. If such a set is found, BCR adds it to $\mathcal{R}$ and continues through the list in search of another overlapping set. After the penultimate edge has been assessed, edges in $\mathcal{R}$ are removed from $\mathcal{L}$ and the next iteration begins. As soon as $\mathcal{R}$ contains overlapping sets that cover all components of $G'$ the process is immediately terminated. The connecting procedure is then used all on sets $R_i \in \mathcal{R}$ to combine the components of $G'$. If, however, no new sets are created during an iteration, and the sets in $\mathcal{R}$ do not cover all components, then it is clear that no more sets exists and thus the components cannot be combined into a single alternating Hamiltonian cycle. Furthermore, if less than two edges remain in $\mathcal{L}$ after an iteration, then no more sets can be created as at least two edges are required to start a new set. 

The connecting procedure operates on a set $R_i$ as follows: for each edge in $R_i$ in turn, the edge from $R \backslash R'$ connecting the lower-indexed vertex of the edge to the higher-indexed vertex of the next edge is added to $R'$ (for the last edge, take the first edge in $R_i$ to be the next edge). The edges in $R_i$ that appear in $R'$ are then removed, resulting in $n+1$ edges in $R'$ that form an alternating Hamiltonian cycle on $G'$ with the edges in $B$. An example of this is shown in Figure~\ref{fig:br}, where BCR has produced a single set $R_1 = \{(v_3, v_{12}),(v_4, v_{11}), (v_5, v_{10})\}$. Given an instance $\mathcal{M}$ of cardinality $n$ of the COP, the Alternating Hamiltonian Construction algorithm is able to determine whether a feasible solution exists, and produce the solution if one does indeed exists, in at most $O(n^2)$ time \cite{hawa2018}.

\begin{figure}[H]	
	\centering
	\begin{subfigure}[h]{0.35\textwidth}
		\includestandalone[width=\textwidth]{figures/bcrandcp}
		%\caption{}
		\label{fig:mbrlist}
	\end{subfigure} \hspace{7mm} %
	\begin{subfigure}[h]{0.25\textwidth}
		\includestandalone[width=\textwidth]{figures/mpsconnect}
		%\caption{}
		\label{fig:mpsconnect}
	\end{subfigure} \hspace{7mm} %
	\begin{subfigure}[h]{0.25\textwidth}
		\includestandalone[width=\textwidth]{figures/mpscycle}
		% \caption{}
		\label{fig:mpscycle}
	\end{subfigure}
	\caption{Bridge-Cover Recognition (BCR) algorithm and connecting procedure used to combine the components of $G'$ to produce a single alternating Hamiltonian cycle that corresponds to a solution $\mathcal{T}$.}
	\label{fig:br}
\end{figure}

% Hidden: Bullet point list for AHC, all \idone
\begin{comment}
{\color{myBlue}
\begin{itemize}[leftmargin=*]
	\item
	\idone{How does this method differ from Becker's?}
	\idone{Exact Polynomial-time algorithm for COP Becker \cite{becker2015} Hawa \cite{hawa2018}.}
	\idone{Model instance graphically, graph $G$ with $V = \{v_1, ...v_{2n}\}$ vertices, one for each value in $\mathcal{M}$ in non-decreasing order.}
	\idone{Preliminary checks using elements, explain, speed.}
	\idone{Cite Lewis \cite{lewis2011} for preliminary checks.}
	\idone{Add two dominating vertices, have value equal to $\tau$, therefore $G$ has $2n+2$ vertices.}
	\idone{Dom vertices will be removed at the end.}
	\idone{degree $2n+1$.}
	\idone{$B$, blue edges between ``partners'' (values that make up a pair in $\mathcal{M}$)}.
	\idone{$|B| = n+1$, perfect matching.}
	\idone{Footnote dom vertices are partners.}
	\idone{Bijective function partners, $p : V \to V$, $p(v_i) = v_j$.}
	\idone{$R$, red edges between vertices whose values add up to $\tau$/meet VSC and are not partners.}
	\idone{Given the bijective function $p:V \to V$ that associates each vertex $v_i \in V$ with its partner $p(v_i)$, we can denote the set of blue edges as $B = \{(v_i, p(v_i)) : v_i \in V\}$.}  
	\idone{Then we have a graph $G = (V, B \cup R)$ (see figure).}
	\idone{Higher-indexed vertices have larger degrees.}
	\idone{Define Ham cycle, formal definition alt Ham cycle.}
	\idone{$R$ = all possible pairings of values.}
	\idone{$B$ cannot be altered, all edges must remain, as each edge represents a pair of values.}
	\idone{Therefore aim is to find set of edges from $R$ that forms alt Ham cycle with $B$.}
	\idone{Ham cycle NP-complete, Karp \cite{karp1972}, alt Ham generalises Ham cycle, Haggkvist \cite{haggkvist1977} proposition, special structure of graphs means can solve in polynomial-time.}
	\idone{MCM - match each vertex with largest possible vertex, create set $R'$.}
	\idone{Cite Mahadev \cite{mahadev1994}.}
	\idone{Explain ``matched'', bijective function.}
	\idone{Can denote $R'$ as $R' = \{(v_i, m(v_i)) : v_i \in V\}$.}
	\idone{Swap of partners.}
	\idone{If $|R'| < n+1$, no solution exists, not enough edges to form cycle, end. Else if $|R'| = n+1$, i.e. matching, then $G' = (V, B \cup R')$ is 2-reg graph, consists of cycles $C_1,...,C_l$.}
	\idone{If $l = 1$, then $G'$ is alt ham cycle, solution found, end.}
	\idone{Else we need to find edges from $R\backslash R'$ that can act as bridges between the components and join them together into a single cycle.}
	\idone{Use BR to find these edges. List edges in order, go through to find edges that meet specific conditions. Continue until penultimate edge reached.} 
	\idone{BR produces sets $R_1, R_2,...$.}
	\idone{If no sets produced, no solution exists, end. Else if $\exists$ set such that $|R_i| = l$, use connecting procedure to join components together, solution found, end.}
	\idone{Else multiple sets need to be used, run MBR to find two or more sets that overlap correctly. If collection $\mathcal{R}^*$ found, connecting procedure on all sets in collection, solution found, end. Else, no solution exists, end.}
	\idone{Guaranteed to find solution, if one exists, in $O(n^2)$ time.}
	\idone{MBR types, check that Beckers method is redundant.}
	\idone{This is called the Alternating Hamiltonian Construction (AHC) algorithm.}
	\idone{BR sorts edges of $R'$.}
\end{itemize}
}
\end{comment}

%--------------------------------------------------------------------------------------
\section{Heuristics for the SCSPP}
\label{sec:scspp}
\noindent A feasible solution for the SCSPP is represented by the set $\mathcal{S} = \{S_1, S_2, ..., S_k\}$ such that
\begin{subequations}
	\begin{alignat}{2}
	\bigcup\nolimits_{i=1}^{k} S_i &= \mathcal{I}, & \label{eqn:packall}\\[3pt]
	S_i \cap S_j &= \emptyset &\quad &\forall \hspace{1mm} i, j \in \{1,2,\dotsc,k\}, \hspace{1mm} i \neq j, \label{eqn:nooverlap} \\[3pt]
	A(S_j) = \sum\nolimits_{i=1}^{|S_j|}w_i &\leq W &\quad &\forall \hspace{1mm} S_j \in \mathcal{S}, \label{eqn:capacity} \\[3pt]
	\textup{\textbf{rhs}}(i) + \textup{\textbf{lhs}}(i+1) &\geq \tau &\quad &\forall \hspace{1mm} i \in \{1, 2,\dotsc,|S_j|-1\}, \hspace{1mm} \forall \hspace{1mm} S_j \in \mathcal{S}. \label{eqn:vscstrip}
	\end{alignat}
\end{subequations}

\noindent As stated in Definition \ref{defn:scspp}, the aim is to minimise $k$. An optimal solution for the SCSPP is a solution consisting of the fewest number of strips $k$ required to feasibly pack all items in $\mathcal{I}$. In this particular MGP, it can be seen that a strip $S_i \in \mathcal{F}$ if the total width of items on the strip does not exceed the strip's capacity \eqref{eqn:capacity} and the vicinal sum constraint is fulfilled \eqref{eqn:vscstrip}. It follows that the SubSCP arises within the SCSPP.

Equations \eqref{eqn:packall}--\eqref{eqn:capacity} are the necessary conditions for the BPP. Although the BPP is simple by definition, finding solutions that meet all conditions can be difficult, and given large instances an exact solution may be impossible to find in a reasonable amount of time. One of the simplest and most well-known heuristics is the First-Fit (FF) heuristic, a greedy online algorithm that packs each item, given in some arbitrary order, into the lowest-indexed bin such that the capacity of the bin is not exceed, opening a new bin when required. It is known that there always exists at least one ordering of the items such that FF produces an optimal solution \cite{lewis2009}. An improvement on FF yields the First-Fit Decreasing (FFD) heuristic, which initially sorts the items in non-increasing order of size. In 2007, it was proven that the worst case for FFD is $\frac{11}{9}k + \frac{6}{9}$, and that this bound is tight \cite{dosa2007}. Due to the initial sorting of the items, the time complexity of FFD is $O(n \lg n)$

For the BPP, a basic lower bound for $k$ is the theoretical minimum, $t = \ceil{\sum_{i=1}^{n} w_i / W}$ \cite{martello1990l}. It can be seen that $t$ will not perform as accurately for the SCSPP as it does for the BPP. Consider a set of $n$ items in which the largest score width $b_i < \tau/2$. Then, there are no pairs of items that can be packed alongside one another feasibly, thus $\mathcal{S}$ will require $n$ strips. The theoretical minimum $t$ does not consider the effect of the minimum scoring distance on the feasibility of the solution. \scspp{If all score widths $> \tau / 2$ then equivalent to BPP and $t$ holds.}

The minimum scoring distance also introduces differences in solutions for the BPP and SCSPP. The obvious difference is that of the ordering and orientation of the items on the strips: unimportant in the BPP, but vital for the feasibility of a solution for the SCSPP. Another distinction arises when attempting to modify solutions. Again, a solution for the BPP remains feasible when an item is removed or a new item is added to a strip (provided the strip can accommodate said item), whereas for the SCSPP this may render a solution infeasible, as the new pairings of score widths may not abide by the vicinal sum constraint. Because of these differences, it is apparent that basic heuristics used for the BPP cannot be used for the SCSPP, as there is no guarantee that the final solution will be feasible. Notice however that these heuristics can still produce feasible, albeit not optimal, solutions for the TPP and CSP-SDCL. An example of this is depicted in Figure~\ref{fig:ffd} where FFD has been used for an instance of the SCSPP, producing invalid alignments of items on three of the strips, thus resulting in an entirely infeasible soltution.

\begin{figure}[H]	
	\centering
	\includestandalone[width=0.45\textwidth]{figures/ffd}
	\caption{Example of an infeasible solution produced using FFD for an instance of the SCSPP. Here, $n = 15$, $W = 1000$, and $\tau = 70$. \scspp{$A(\mathcal{I}) = 5268$, therefore $t = 6$.}}	
	\label{fig:ffd}
\end{figure}

\noindent Since the SCSPP generalises the BPP, it follows that the SCSPP is also NP-hard \cite{garey1979}. Assuming $P \neq NP$, we cannot hope to find an optimal solution for all instances of the SCSPP in polynomial time. The simplest methods to implement are heuristics, which trade optimality for speed. As the SCSPP is a relatively new problem, few methods have been seen in literature. Some basic heuristics were introduced in Hawa et al. \cite{hawa2018}, two of which are based on the FFD heuristic for the BPP. The first, named the Modified First-Fit Decreasing (MFFD) heuristic, performs in the same fashion as FFD, with the extra condition that an item can only be packed onto the end of a strip if the score width on the end of the strip and one of the score widths on the item meet the vicinal sum constraint. An improvement on MFFD is their second heuristic, MFFD$^+$, which incorporates AHC. Rather than attempting to pack each item onto the end of the strips, MFFD$^+$ calls upon AHC to find a feasible ordering of all items on a given strip that includes the item to be packed.\footnote{If an item is the first to be packed on a strip, it is placed in a regular orientation, $(a_i, b_i)$.} Clearly, MFFD$^+$ is the superior of the two, as the application of AHC guarantees that a feasible ordering will be found if it exists. The limitation on MFFD of only packing items on the end has the potential to increase the number of strips being used in a final solution unnecessarily. Figure~\ref{fig:mffdvsmffdplus} compares solutions produced using MFFD and MFFD$^+$ using the same instance of the SCSPP as seen in Figure~\ref{fig:ffd}. Note how MFFD$^+$ is able to form a solution using the same number of strips as FFD while still satisfying the vicinal sum constraint.

\begin{figure}[H]	
	\centering
	\begin{subfigure}[h]{0.45\textwidth}
		\includestandalone[width=\textwidth]{figures/mffd}
		\caption{MFFD}
		\label{fig:mffd}
	\end{subfigure} \hspace{10mm}
	\begin{subfigure}[h]{0.45\textwidth}
		\includestandalone[width=\textwidth]{figures/mffdplus}
		\caption{MFFD$^+$}
		\label{fig:mffdplus}
	\end{subfigure}
	\caption{Solutions produced using MFFD and MFFD$^+$ using the same instance as in the previous figure.}
	\label{fig:mffdvsmffdplus}
\end{figure}

% Hidden: Bullet point list for SCSPP, all \idone 
\begin{comment}
{\color{myPink}
\begin{itemize}[leftmargin=*]
	\item
	\idone{Feasible solution $\mathcal{S} = \{S_1,...,S_k\}$ and four conditions.}
	\idone{SCSPP at least as hard as BPP, NP-hard (SCSPP generalises BPP, assuming $P \neq NP$).}
	\idone{Cannot find solution in reasonable time, therefore use heuristics, near optimal solutions faster.}
	\idone{Lit review, heuristics used for BPP/SPP, FFD, BFD etc., how do these work with SCSPP? State how they could work with TPP/SDCL because of what was mentioned in introduction, but cannot guarantee feas solution for SCSPP due to VSC.}
	\idone{Optimal solution requires fewest number of strips $k$ to pack all items.}
	\idone{Lower bound for $k$ is theoretical minimum $t = \ceil*{\sum_{i=1}^{n}w_i / W}$.}
	\idone{Lowerbound cite Martello \cite{martello1990l}.}
	\idone{$t$ not accurate for SCSPP, doesn't take into account $\tau$, if $\tau >$ 2x largest score width then $n$ strips required, one for each item, regardless of item widths or strip width $W$.}
	\idone{May find optimal solution without knowing, as optimal number of strips for a given problem instance of the SCSPP may be greater than $t$.}
	\idone{Can we show a solution is optimal? If not, why?}
	\idone{Differences BPP and SCSPP: (1) Importance of order and orientation of items; (2) Feasibility of strips when adding/removing/rearranging items.}
	\idone{$\mathcal{F}$ denotes the set of all subsets $\mathcal{I}' \subseteq \mathcal{I}$ of items that can be feasibly packed onto a single strip, i.e. $\sum_{i \in S} w_i \leq W$ and VSC met.}
	\idone{Heuristics for SCSPP Hawa \cite{hawa2018}.} 
	\idone{Our implementation of AHC in MFFD$^+$ has the preliminary check, previous paper does not.}
	\idone{MFFD - FFD but only place item if score width on end of strip meets VSC with a score width on the item (check both orientations) - operates in the same way as FFD, except additional step of checking that VSC is met with one of the score widths of the item and the right-most score width on the strip, i.e. check that $S \cup \{i\} \in \mathcal{F}$.}
	\idone{MFFD$^+$ - FFD, incorporates AHC, instead of checking VSC on end of strip, run AHC using all items on strip and current item. If item fits on strip, run AHC on items on strip and current item. If solution found, replace current arrangement of items on strip with arrangement found by AHC including new item, else move on to next strip.}
	\idone{If item starts new strip, pack item in regular orientation, i.e. $(a_i, b_i)$, smaller score width on left-hand side.}
	\idone{ MFFD$^+$ can find solution for subset of items, rearrange all items, MFFD cannot, only place items on end.}
	\idone{MFFD may require extra strips for items even though feasible ordering exists using AHC.}
\end{itemize}
}
\end{comment}

\subsection{Experimental Results - Heuristics}
\label{sub:expheuristics}
\noindent Although a comparison of these heuristics is given in \cite{hawa2018}, we performed out own tests where AHC used in MFFD$^+$ includes the preliminary check and the new BCR procedure. For our experiments, we produced two different types of instances: ``artificial'', in which the items are strongly heterogeneous (the items have varying widths and score widths); and ``real'', where items are weakly hetergeneous (many items have the same dimensions). For each type 1000 instances were generated using sets of 100, 500, and 1000 items, giving a total of 6000 problem instances. The minimum scoring distance $\tau$ was set to 70mm for all instances - the industry standard. All items have widths $w_i \in [150,1000]$ and score widths $a_i, b_i \in [1,70]$ selected uniform randomly, and equal height $H=1$. Two different strips widths, $W = 2500$ and 5000 were used to alter the number of items per strip. As optimal solutions are not available, we calculated the solution quality $q = |\mathcal{S}|/ t$ which compares each solution to the theoretical minimum. Table~\ref{table:heuristics} contains the results from the experiments. All instances were completed in under 125ms.

\begin{table}[h!]
\centering
\caption{MFFD vs MFFD$^+$}
\begin{threeparttable}
\begin{tabular}{c@{\hspace{20pt}}c@{\hspace{20pt}}c@{\hspace{20pt}}c@{\hspace{20pt}}c@{\hspace{20pt}}ccc@{\hspace{20pt}}c@{\hspace{20pt}}c@{\hspace{8pt}}}\toprule
	& & & \multicolumn{3}{c}{MFFD} &\phantom{a}& \multicolumn{3}{c}{MFFD$^+$}\\
	\cmidrule{4-6} \cmidrule{8-10}
	Type, $n$ & $W$ & $t$\tnote{1} & $|\mathcal{S}|$\tnote{2} & $\# t$\tnote{3} & $q$\tnote{4} && $|\mathcal{S}|$ & $\# t$ & $q$\\ \midrule	
	a, 100 & 2500 & 23.323 & 30.754 & 0 & 1.320 && 28.457 & 26 & 1.221 \\
	a, 100 & 5000 & 11.922 & 23.583 & 0 & 1.982 && 19.881 & 7 & 1.670  \\
	\midrule
	a, 500 & 2500 & 114.942 & 140.206 & 0 & 1.220 && 132.647 & 0 & 1.154 \\
	a, 500 & 5000 & 57.722 & 103.209 & 0 & 1.789 && 89.544 & 0 & 1.552 \\
	\midrule
	a, 1000 & 2500 & 229.437 & 271.919 & 0 & 1.185 && 258.388 & 0 & 1.126 \\
	a, 1000 & 5000 & 114.965 & 198.325 & 0 & 1.725 && 172.613 & 0 & 1.502 \\
	\midrule
	\midrule
	r, 100 & 2500 & 23.473 & 37.069 & 5 & 1.600 && 35.419 & 16 & 1.523 \\
	r, 100 & 5000 & 11.981 & 32.348 & 1 & 2.731 && 29.611 & 5 & 2.497 \\
	\midrule
	r, 500 & 2500 & 115.239 & 184.106 & 0 & 1.612 && 177.249 & 0 & 1.551 \\
	r, 500 & 5000 & 57.865 & 163.819 & 0 & 2.860 && 153.416 & 0 & 2.678 \\
	\midrule
	r, 1000 & 2500 & 229.946 & 368.453 & 0 & 1.617 && 355.042 & 0 & 1.557 \\
	r, 1000 & 5000 & 115.227 & 328.618 & 0 & 2.882 && 308.642 & 0 & 2.706 \\
	\bottomrule
\end{tabular}
\vspace{0.2cm} %
\begin{tablenotes}
	\item[1] $t = \ceil{\sum_{i=1}^{n} w_i / W}$ (mean from 1000 instances).
	\item[2] Number of strips per solution (mean from 1000 instances).
	\item[3] Number of instances where a solution using $t$ strips was found (from 1000 instances).
	\item[4] $q = |\mathcal{S}| /t$ (mean from 1000 instances).
\end{tablenotes}	
\end{threeparttable}	
\label{table:heuristics}
\end{table}

It can be seen that MFFD$^+$ produces solutions using fewer strips on average than MFFD for all 12 instance classes. MFFD was only able to find solutions using $t$ strips for two instance classes, whereas MFFD$^+$ produced solutions using $t$ strips for four instances classes. Solutions using fewer strips were found less frequently on average when the strip width is larger. The solution quality $q$ for the six instance classes where $W = 5000$ is significantly higher than when $W = 2500$. A larger strip width means the strips can accommodate more items, however this increases the number of adjacent score widths that must fulfil the vicinal sum constraint. Note also that $q$ is higher for real instances in comparison to artificial instances, as due to the reduced variety of item widths and score widths it is difficult to attain feasible packings. 

Indeed, it may be that optimal solutions were found in many of these instances, as the lower bound for $k$ may actually be higher than the theoretical minimum calculated. These experiments provide clear results that the use of an exact polynomial-time algorithm, AHC, within a heuristic is more powerful than using a simple heuristic alone. In spite of this, MFFD$^+$ lacks the capability of rearranging items \emph{between} strips, that is, once an item has been packed onto a strip, it must stay on that strip (although the position of the item on the strip can change due to the use of AHC). It follows that every item packed has an affect on where the unplaced items can be placed, i.e. where an item is packed is dependent on where previous items have been packed. These limitations associated with greedy heuristics lead us to explore other superior methods.

% Hidden: Bullet point list for Heuristics Experiments, all \idone
\begin{comment}
{\color{myPink}
\begin{itemize}[leftmargin=*]
	\item
	\idone{Artificial - items are strongly heterogeneous (different widths/score widths).}
	\idone{Real - items are weakly heterogeneous (many items have the same score widths/widths).}
	\idone{Solution quality $q = |\mathcal{S}|/ t$.}
	\idone{$t$, $q$, and $|\mathcal{S}|$ are all means from 1000 instances}
	\idone{Remove fitness from table.}
	\idone{MFFD only finds solutions == $t$ for real, 100, $W=2500$ (5 instances) and $W=5000$ (1 instance).}
	\idone{MFFD$^+$ find solutions == $t$ for artificial, 100, $W = 2500$ (26 instances) and $W=5000$ (7 instances), and real, 100, $W=2500$ (16 instances) amd $W=5000$ (5 instances).}
	\idone{Solution quality of MFFD$^+$ better (that is, closer to 1) than MFFD for every problem type on average.}
	\idone{Instances where $|\mathcal{S}|$ == $t$ know that opt solution found.}
	\idone{Could be instances where opt solution found but we are unable to show that it is optimal.}
	\idone{MFFD$^+$ slower, for each item AHC could be executed up to $|\mathcal{S}|$ times.}
	\idone{Running experiments again, this uses BCR and preliminary checks, unlike previous paper. }
	\idone{Table shows...}
	\idone{Time it takes to run heuristics.}
	\idone{State that there are 20 types on average for real instances (no longer in table).}
	\idone{MFFD$^+$ slower, could be a different ordering of items works better (rather than items in decreasing order of widths), one item at a time might not find feasible ordering, may be that feasible ordering exists when packing multiple items simultaneously.}
	\idone{Fast greedy heuristics not as good, other methods superior, therefore use evolutionary algorithm (lead on to next section).}
\end{itemize}
}
\end{comment}

%--------------------------------------------------------------------------------------
\section{Evolutionary Algorithm}
\label{sec:ea}
\noindent We now introduce an evolutionary algorithm (EA) for the SCSPP to improve on the results of the previous heuristics. An evolutionary algorithm is a metaheuristic optimisation algorithm inspired by the natural evolutionary process. Candidate solutions to the problem form the initial population, and procedures emulating selection, reproduction, recombination and mutation are used to create the next generation of solutions. This iterated process results in the evolution of the solutions. Each solution is evaluated based on a given critera, and individuals which are more suited to the environment are given more oppportunity to breed while those which are less so are eliminated. EAs have been used for a variety of grouping problems with positive results \cite{lewis2017, falkenauer1996, quiroz2015}. \ea{Iterative procedure, plus local search and xover, swapping items between strips, better than previous heuristics as repetition means more possible item combinations on strips, EA has the ability to move items between strips.}

Within the EA framework, we investigate three different recombination operators as well as a local search procedure inspired by Martello and Toth \cite{martello1990l}. AHC is also integrated into the EA to solve instances of the SubSCP. Throughout the EA we maintain two sets, $\mathcal{A}$ and $\mathcal{B}$, containing subsets of items that AHC has determined produce feasible or infeasible orderings, respectively. When an instance of the subproblem occurs, these sets are searched before calling AHC; hence AHC is executed at most once for each distinct subset of items.

% Hidden: Bullet point list for EA intro, all \idone
\begin{comment}
{\color{myRed}
\begin{itemize}[leftmargin=*]
	\item
	\idone{Why use an EA? Lit review of EAs in BPP problems, benefits (repeated multiple times, best characteristics of strips/solutions).}
	\idone{Whenever subproblem occurs, AHC will be used.}
	\idone{Sets $\mathcal{A}$ and $\mathcal{B}$, subsets of items that produce feasible/infeasible orderings determined by AHC. Search these sets before calling AHC, therefore AHC used at most once for each distinct subset of items $\mathcal{I}' \subseteq \mathcal{I}$.}
	\idone{Feasible solution $\mathcal{S}$ means solution abides by all constraints.}
\end{itemize}
}
\end{comment}

\subsection{Recombination}
\label{sub:xover}
\noindent A recombination operator is used to generate new solutions from an existing population. This is done by taking two parent solutions, $\mathcal{S}_1$ and $\mathcal{S}_2$, and combining them to create a new offspring solution. The operator determines which characteristics from each parent should be inherited by the offspring. Thus, recombination attempts to retain the best elements of each parent, with the aim of producing a superior offspring. As described in the previous section, we cannot simply copy individual items from each parent into the offspring because the vicinal sum constraint may not be satisfied. Therefore, the operators used in this EA have been designed to ensure offspring feasibility.

The first operator is based on the grouping genetic algorithm (GGA) of Falkenauer \cite{falkenauer1992}. The strips of the second parent solution $\mathcal{S}_2$ are permuted, and two strips $S_i$ and $S_j$ from $\mathcal{S}_2$ are selected randomly (where $1 \leq i < j \leq |\mathcal{S}_2|$). All strips between and including $S_i$ and $S_j$ are inserted into an offspring solution $\mathcal{S}$. GGA then adds to the offspring all strips from $\mathcal{S}_1$ that do not contain items already present in the offspring. Note that the strips chosen from $\mathcal{S}_2$ cannot both be the outermost strips, that is, GGA cannot choose both $i = 1$ \emph{and} $j = |\mathcal{S}_2|$, as doing so would result in all strips from $\mathcal{S}_2$ being copied into the offspring, preventing the addition of any strips from $\mathcal{S}_1$. \ea{What is GGA's aim?}

Another operator implemented is the alternating grouping crossover (AGX), and is analogous to that of Quiroz-Castellanos et al. \cite{quiroz2015}. Starting with the parent solution containing the fullest strip, AGX inserts this strip into an offspring solution $\mathcal{S}$, and strips containing items in $\mathcal{S}$ are removed from the other parent.\footnote{For both AGX and AGX', in the event that both parents contain strips with equal maximum fullness/number of items, the initial parent solution is chosen at random.} The operator then inserts the fullest strip from the modified parent into $\mathcal{S}$ and removes strips from the first parent. AGX continues to alternate between parents, selecting the fullest strip $max_{S_j \in \mathcal{S}_p} (A(S_j))$ (where $\mathcal{S}_p$ is the parent solution under consideration, $p \in \{1,2\}$) until $min (|\mathcal{S}_1|,|\mathcal{S}_2|) - 1$ strips have been added to the offspring solution. \ea{How is AGX useful?}

Our final operator, AGX$'$, performs in a similar manner to AGX, however rather than choosing the fullest strip to insert into the offspring solution AGX$'$ selects strips containing the most items, $max_{S_j \in \mathcal{S}_p} (|S_j|)$. This method has the ability to preserve strips containing items that are harder to pack along with other items. 

In order to maintain feasibility, the operators remove entire strips containing duplicate items, rather than individual items. These strips may also contain items that are not present in the offspring solution. Consequently, on completion of the crossover, the offspring solution $\mathcal{S}$ may not contain all $n$ items, and is therefore not yet a full solution. To rectify this, MFFD$^+$ is applied using the missing items to form a partial solution $\mathcal{S}^*$. The partial offspring solution $\mathcal{S}$ and $\mathcal{S}^*$ are then used as input into a local search procedure to create a full feasible offspring solution.

% Hidden: Bullet point list for EA xOver, all \idone
\begin{comment}
{\color{myRed}
\begin{itemize}[leftmargin=*]
	\item
	\idone{Purpose of recombination/crossover, select best characteristics of each parent solution to form offspring.} 
	\idone{Prevent duplicate items.}
	\idone{Note the differences between BPP and SCSPP, cannot simply remove individual items from strips, may cause the arrangement of remaining items to become infeasible (VSC not met between items).}
	\idone{GGA Falkenauer \cite{falkenauer1992}, permute strips of $\mathcal{S}_2$, select two strips, take all strips between and including these strips and put into offspring solution $\mathcal{S}$. Add from $\mathcal{S}_1$ strips that do not contain items already in offspring. Note, choose strips $1 \leq i < j \leq |\mathcal{S}_2|$ \textbf{and} cannot have $i = 1 \land j = |\mathcal{S}_2|$ unlike in Rhyd paper - ensures that at least 2 strips from $\mathcal{S}_2$ are in offspring, and prevents all strips from $\mathcal{S}_2$ from being selected, as this would mean that all strips from $\mathcal{S}_1$ would be deleted from the offspring, and so offspring will not have strips from both parents. Ensures that offspring has characteristics from both parents.}
	\idone{GPX', Quiroz \cite{quiroz2015} (check) - Alternate between parent solutions, add fullest strip to offspring, delete strips from other parent containing items in offspring. Continue until $min(|\mathcal{S}_1|,|\mathcal{S}_2|)-1$ strips in offspring. Initially, if both parents have fullest strip, break ties randomly.}
	\idone{GPN - Operates in the same manner as GPX', except chose strip containing the most items.} 
	\idone{Initially, if both parents have strip containing same number of max items, choose the parent whose strip is the fullest. If equal for both strips, choose randomly.} 
	\idone{What does each xOver aim to do? Prevent duplicates (by deleting strips from other parent).}
	\idone{Missing items, apply MFFD$^+$ to produce partial solution $\mathcal{S}_Y$, then apply Local search (described in next section) on offspring and $\mathcal{S}_Y$ to produce full offspring $\mathcal{S}$}
	\idone{Many missing items in comparison to BPP as cannot delete individual items due to VSC, have to delete entire strips, could contain items not yet in offspring.}
\end{itemize}
}
\end{comment}

\subsection{Local Search}
\label{sub:localsearch}
\noindent Our local search method takes in two feasible partial solutions, $\mathcal{S}$ and $\mathcal{S}^*$, permutes the strips of both, and then attempts to move items between the two solutions in four stages: 
\begin{enumerate*}[label={(\roman*)}]
	\item swapping a pair of items from a strip in $\mathcal{S}$ with a pair of items from a strip in $\mathcal{S}^*$;\label{item:pairpair}
	\item swapping a pair of items from a strip in $\mathcal{S}$ with an individual item from a strip in $\mathcal{S}^*$;\label{item:pairsin}
	\item swapping individual items from strips in $\mathcal{S}$ and $\mathcal{S}^*$;\label{item:sinsin} and
	\item moving an item from a strip in $\mathcal{S}^*$ to a strip in $\mathcal{S}$.\label{item:movesin}
\end{enumerate*} 
During stages \ref{item:pairpair}--\ref{item:sinsin}, the width of the item(s) from $\mathcal{S}^*$ must exceed the width of the item(s) from $\mathcal{S}$. Once a swap or move has been perfomed, the procedure immediately moves on to the next stage. This method is repeated until all four stages have been executed in succession with no changes to $\mathcal{S}$ or $\mathcal{S}^*$. Then, MFFD$^+$ is applied to any items remaining in $\mathcal{S}^*$, generating a new feasible partial solution $\mathcal{S}^{**}$, with $|\mathcal{S}^{**}| \leq |\mathcal{S}^*|$. The strips in $\mathcal{S}^{**}$ are then inserted into $\mathcal{S}$ to form a full feasible solution.

This method is based on the dominance criterion of Martello and Toth \cite{martello1990l}: if a strip $S_x$ \emph{dominates} a strip $S_y$, then a solution containing $S_x$ will have no more strips than a solution containing $S_y$. The local search procedure is in fact a local search for dominating strips. By moving larger items into $\mathcal{S}$, the fullness $A(S_j)$ of the strips  $S_j \in \mathcal{S}$ increases whilst the number of items per strip is maintained or decreases, improving the \ea{quality} of the strips in $\mathcal{S}$. Simultaneously, items moved into $\mathcal{S}^*$ are smaller and therefore easier to repack onto strips in $\mathcal{S}$ during stage~\ref{item:movesin}. Variations of this method can be seen in \cite{lewis2009, lewis2017, falkenauer1996, levine2004}, however the addition of the vicinal sum constraint results in fewer changes than seen in these previous implementations. By iterating the stages, numerous distinct subsets of items on the strips are produced, generating more possibilities for feasible orderings of items.

% Hidden: Bullet point list for EA Local Search, all \idone
\begin{comment}
{\color{myRed}
\begin{itemize}[leftmargin=*]
	\item
	\idone{Purpose of local search, takes two partial solutions $\mathcal{S}_X$ and $\mathcal{S}_Y$.}
	\idone{Initially permute strips.}
	\idone{4 stages, 3 swap, 1 move. width of items from $\mathcal{S}_X$ must be less than width of items from $\mathcal{S}_Y$ in first three stages.}
	\idone{Moves smaller items into $\mathcal{S}_Y$, larger items into $\mathcal{S}_X$.}
	\idone{If swap/move performed, move onto next stage. Repeat until no changes to either partial solution.}
	\idone{MFFD$^+$ applied on items in $\mathcal{S}_Y$ to produce partial solution $\mathcal{S}_Y'$. Then strips from $\mathcal{S}_X$ and $\mathcal{S}_Y'$ inserted into $\mathcal{S}$ (this is not necessarily offspring).}
	\idone{Based on dominance criterion of Martello and Toth \cite{martello1990l}.}
	\idone{If strip $S_1$ dominates $S_2$, the a solution using $S_1$ will have no more strips than a solution using $S_2$.}
	\idone{Aim - increase fullness of strips in $\mathcal{S}_X$ while maintaining or decreasing number of items on strips.}
	\idone{Moves smaller items into $\mathcal{S}_Y$, easier to repack in stage (4).}
	\idone{Variations of procedure seen in Falkenauer \cite{falkenauer1996}, Lewis \cite{lewis2009} \cite{lewis2017}, Levine \cite{levine2004}.}
	\idone{VSC makes swaps/moves more difficult, this means local search takes longer, fewer swaps than seen in \cite{lewis2009}, \cite{lewis2017}, \cite{levine2004}, \cite{falkenauer1996} (check).}
	\idone{Repeating process produces different combinations of items, more possibilities to find feasible ordering.}
	\idone{Well-filled strips with larger items preferable over less-filled strips with smaller items (Levine \cite{levine2004}).}
\end{itemize}
}
\end{comment}

\subsection{Evolutionary Algorithm Framework}
\label{sub:eaframework}
\noindent Our EA for the SCSPP begins by producing candidate solutions to form an initial population, with one solution created using MFFD$^+$ and the rest using MFFR$^+$ (the same method as MFFD$^+$ with the items in a random order). Each solution is mutated before being inserted into the population. The mutation of a solution $\mathcal{S}$ involves permuting the strips and inserting $1 < r < |\mathcal{S}|$ strips selected randomly from $\mathcal{S}$ into a set $\mathcal{S}^*$. Local search is then executed using these two partial solutions to produce a full feasible solution $\mathcal{S}$. 

An iteration of the EA \ea{involves} selecting two parent solutions, $\mathcal{S}_1$ and $\mathcal{S}_2$, from the population at random, applying a recombination operator to produce an offspring solution $\mathcal{S}$, then finally mutating $\mathcal{S}$ before replacing the least fit of the two parents in the population.

A tailored function is used to determine the fitness of a solution, as opposed to simply relying on the number of strips within a solution. The reason for this is two-fold: firstly, given two solutions of equal size, it is impossible to determine the fitter solution based on the number of strips alone. Secondly, we note that the fitness of a solution not only depends on the number of strips used, but also \emph{how} the items are packed onto the strips. It is clear that if the strips' capacities are taken advantage of, fewer strips will be needed to pack all items, compared to if the strips are only half full, requiring more strips than necessary to accommodate all of the items. A solution comprising fuller strips may also contain strips that are nearly empty, which is beneficial as it allows futher items to be packed, or the residual material could be used for other means. 

We therefore make use of the following function to calculate the fitness of a solution $\mathcal{S}$ \cite{falkenauer1992}:
\begin{equation}
	f(\mathcal{S}) = \frac{\sum_{S_j \in \mathcal{S}} (A(S_j)/W)^2}{|\mathcal{S}|}
\end{equation}

\noindent which assigns higher values to fitter solutions. This function exerts evolutionary pressure by favouring solutions with superior parkings, reducing the ability for less fit solutions to reproduce. Note also that if $|\mathcal{S}_1| < |\mathcal{S}_2|$  then $f(\mathcal{S}_1) > f(\mathcal{S}_2)$, hence a global optimum for this fitness function is associated with the optimal solution containing the fewest number of strips.

It is also important to consider how the vicinal sum constraint affects the fitness of a solution. Given two half-full strips containing items from an instance of the classical BPP, it may be possible to move items betweenthe strips such that one strip is almost full and the other nearly empty. Now, suppose the items inthe half-filled strips have score lines, and are from an instance of the SCSPP. Obviously, it is more difficult to shuffle the items between the strips as the vicinal sum constraint may be violated. Thus, fuller strips in the SCSPP are extremely valuable, and the fitness function ensures that solutions containing fuller strips are preserved in the population throughout generations.

% Hidden: Bullet point list for EA Framework, all \idone
\begin{comment}
{\color{myRed}
\begin{itemize}[leftmargin=*]
	\item
	\idone{Create candidate solutions for initial population, one solution using MFFD$^+$, the others using MFFR$^+$ (items sorted in random order as opposed to decreasing order of widths), each solution is mutated before being inserted into the population.}
	\idone{Each iteration of EA, two parent solutions $\mathcal{S}_1$ and $\mathcal{S}_2$ are chosen from the population at random, crossover operator used to create single offspring solution $\mathcal{S}$, which is then mutated before replacing the least fit parent solution in the population (breaking ties randomly).}
	\idone{Fitness function $f(\mathcal{S}) = \frac{\sum_{S \in \mathcal{S}} (A(S)/W)^2}{|\mathcal{S}|}$ \cite{falkenauer1992}.}
	\idone{$A(S) = \sum_{i \in S} w_i$ - ``fullness''/total width of all items on strip $S$}
	\idone{value between 0 and 1, closer to 1 = better packing.}
	\idone{Although two solutions $\mathcal{S}_1$ and $\mathcal{S}_2$ may contain the same number of strips/have the same size, fitness value identifies which solution contains a better packing of the items - distinguish between two solutions of the same size.}
	\idone{Not only considering how many strips required to pack all items, but how those items are packed onto the strips (optimise how the strips are packed).}
	\idone{Better packed strips means full capacity of strip is taken advantage of, reduce waste material on strips, plus nearly empty strips can be used to accommodate more items, or waste material can be reused.}
	\idone{If the strips are packed well, fewer strips required, else if strips are not packed well, extra strips required to pack items that cannot be packed onto strips already being used.}
	\idone{Take into account vicinal sum constraint - there may be two strips that are roughly equally filled, however even though it may seem that the items could be shuffled so that more are on one strip (thus a better packing), it may not be possible due to the constraint - not taken into account in other applications of this fitness value, however will still use this fitness value as our aim is to produce the best packing possible given the constraint.}
	\idone{As an example, use fitness values from example solutions created using FFD/MFFD/MFFD$^+$ from Heuristics section figures.}
\end{itemize}
}
\end{comment}

\subsection{Experimental Results - EA}
\label{sub:expea}
\noindent To allow for a fair comparison with the heuristics in Section~\ref{sec:scspp}, the same six sets of problem instances from the previous experiments were used to obtain results from our EA. Each of the instance classes were executed six times using different recombination operators and strip widths, produced 36 subclasses overall. We settled for an initial population containing 25 candidate solutions, which was shown to produce the best results. The advantage of using a small population size was to be expected, as the vicinal sum constaint limits the number of feasible candidiate solutions using a reasonable number of strips. It was seen that larger population sizes contained an abundance of poor solutions comprising over twice the number of strips as used in the final solution, whilst even smaller populations lacked the diversity needed to escape local optima. Across all instances, the EA had a fixed time limit of 300 seconds. Table~\ref{table:ea} displays the results obtained from the experiments.

\begin{table}[h!]
\centering
\caption{\ea{EA comparisons}}
\begin{threeparttable}
\begin{tabular}{c@{\hspace{15pt}}c@{\hspace{15pt}}c@{\hspace{15pt}}c@{\hspace{15pt}}c@{\hspace{15pt}}ccc@{\hspace{15pt}}c@{\hspace{15pt}}ccc@{\hspace{15pt}}c@{\hspace{15pt}}c}\toprule
	& & & \multicolumn{3}{c}{GGA} &\phantom{a}& \multicolumn{3}{c}{AGX} &\phantom{a}& \multicolumn{3}{c}{AGX$'$}\\
	\cmidrule{4-6} \cmidrule{8-10} \cmidrule{12-14}
	Type, $n$& $W$ & $t$\tnote{1} & $|\mathcal{S}|$\tnote{2} & $\# t$\tnote{3} & $q$\tnote{4} && $|\mathcal{S}|$ & $\# t$ & $q$ && $|\mathcal{S}|$ & $\# t$ & $q$\\ \midrule \midrule
	a, 100 & 2500 & 23.323 & EA121 & - & - && EA122 & - & - && EA123 & - & -\\
	a, 100 & 5000 & 11.922 & EA151 & - & - && EA152 & - & - && EA153 & - & -\\
	\midrule
	a, 500 & 2500 & 114.942 & EA521 & - & - && EA522 & - & - && EA523 & - & -\\
	a, 500 & 5000 & 57.722 & EA551 & - & - && EA552 & - & - && EA553 & - & -\\
	\midrule
	a, 1000 & 2500 & 229.437 & EA1021 & - & - && EA1022 & - & - && EA1023 & - & - \\
	a, 1000 & 5000 & 114.965 & ER1051 & - & - && EA1052 & - & - && EA1053 & - & - \\
	\midrule \midrule
	r, 100 & 2500 & 23.473 & ER121 & - & - && ER122 & - & - && ER123 & - & - \\
	r, 100 & 5000 & 11.981 & ER151 & - & - && ER152 & - & - && ER153 & - & - \\
	\midrule
	r, 500 & 2500 & 115.239 & ER521 & - & - && ER522 & - & - && ER523 & - & - \\
	r, 500 & 5000 & 57.865 & ER551 & - & - && ER552 & - & - && ER553 & - & - \\
	\midrule
	r, 1000 & 2500 & 229.946 & ER1021 & - & - && ER1022 & - & - && ER1023 & - & - \\
	r, 1000 & 5000 & 115.227 & ER1051 & - & - && ER1052 & - & - && ER1053 & - & - \\
	\bottomrule
\end{tabular}	
\vspace{0.2cm} %
\begin{tablenotes}
	\item[1] $t = \ceil{\sum_{i=1}^{n} w_i / W}$ (mean from 1000 instances).
	\item[2] Number of strips per solution (mean from 1000 instances).
	\item[3] Number of instances where a solution using $t$ strips was found (from 1000 instances).
	\item[4] $q = |\mathcal{S}| /t$ (mean from 1000 instances).
\end{tablenotes}
\end{threeparttable}
\label{table:ea}
\end{table}

{\color{myRed}
\begin{itemize}[leftmargin=*]
	\item Same 6 sets of instances used for heuristics used for EA.
	\item Run each set 6 times, different combinations of parameters, GGA/GPX'/GPN and strip width 2500 and 5000, 36 combinations total
	\item In EA output file to compare:
	\begin{itemize}[leftmargin=*]
		\item Proportion of times feasPacking/infeasPacking sets used instead of AHC
		\item Number of strips in best solution found
		\item Number of iterations of EA within the given time limit
		\item Fitness value of the best solution found
		\item solution quality $q = |\mathcal{S}|/t$ (=1 if lower bound met).
	\end{itemize}
	\item At the end of each EA iteration, add all strips from offspring to feasPacking set to use in post optimisation phase.
	\item Number of solution in initial population = 25.
	\item Time limit for EA - 300 seconds.
	\item Time graph output.
\end{itemize}
}

%--------------------------------------------------------------------------------------
\section{Postoptimisation}
\label{sec:postopt}
\noindent Given a collection $\mathcal{S}$ of subsets of a set $X$, an \emph{exact cover} is a subcollection $\mathcal{S}^*$ of $\mathcal{S}$ such that each element in $X$ is contained in exactly one subset in $\mathcal{S}^*$.

Consider an $m\times n$ binary matrix $X$. Let $M = \{1,2,\dotsc,m\}$ and $N = \{1,2,\dotsc,n\}$ be the rows and columns of the matrix respectively. Now, suppose each row $i \in M$ represents a strip, and each column $j \in N$ represents an item. Then, an element of the matrix $x_{ij} = 1$ iff item $j$ is on strip $i$. It can be said in this case that row $i$ \emph{covers} column $j$. Thus, the problem is to find a minimum cardinality subset of rows $S \subseteq M$ such that each column $j \in N$ is covered by exactly one row $i \in S$. This problem can be formulated as the following integer linear program.

\begin{subequations}
	\begin{alignat}{3}
		\text{minimise  } &\sum_{i \in M} c_i & \\[3pt]
		\text{subject to  } &\sum_{i \in M} x_{ij} c_i = 1 &\quad &\forall \hspace{1mm} j \in N \\[3pt]
		&c_i \in \{0,1\} & &\forall \hspace{1mm} i \in M
	\end{alignat}
\end{subequations}

\[c_i =
\begin{cases} 
1 & \text{if } i \in S \\
0 & \text{otherwise} 
\end{cases}
\]

\noindent The exact cover problem, determining whether a subcollection $\mathcal{S}^*$ exists, is a decision problem, and one of Karp's 21 NP-complete problems \cite{karp1972}. However, if we know that an exact cover does indeed exist, we can alter the problem to instead find the smallest subcollection.

Given a set of feasible strips, we can use this to find a solution to the SCSPP. Since the strips provided are already feasible, the complications associated with the vicinal sum constraint are eliminated.

One method of solving the exact cover problem is by using a recursive depth-first backtracking algorithm, dubbed ``Algorithm X'' by Donald Knuth \cite{knuth2000}, which is used to find all solutions. Given our problem, it is unecessary to find all solutions. Instead, we have adapted the procedure to only seach for solutions that improve upon the best solution found thus far. Ideally, we would use the entire set of feasible strips $\mathcal{F}$ to form the matrix, however this could include hundreds of millions of feasible strips. Instead, we chose to use the set $\mathcal{A}$ created during the EA, which contains feasible strips found during the algorithm.

{\color{myGreen}
\begin{itemize}[leftmargin=*]
	\item Cite Malaguti \cite{malaguti2008}.
	\item Exact cover formulation, NP-hard, state the IP, describe DLX.
	\item Cite Knuth \cite{knuth2000} dancing steps.
	\item Use strips in feasPacking for post opt.
	\item Compare with EA output - is a better solution found, or a solution with the same number of strips but a better fitness value?
	\item Is the post opt phase able to find a solution equal to the lowerbound?
	\item Post opt will only ever find a solution equal to or better than the solution found in EA, never worse.
	\item Execution time.
	\item Set cover problem is optimisation problem, find min number of sets.
	\item Exact cover problem is decision problem, does a set exist.
	\item However since we add every item on its own strip to feasPacking, we know that a set exists.
	\item Problem is to find minimum number of strips that covers all items and contains every item exactly once.
	\idone{$X = (x_{ij})$ - $m$ x $n$ matrix (previously matrix $A = (a_{ij})$).}
	\idone{$M = \{1, 2,\dotsc, m\}$ - rows of the matrix, each row $i \in M$ is a strip.}
	\idone{$N = \{1, 2,\dotsc,n\}$ - columns of the matrix, each column $j \in N$ is an item.}
	\idone{$x_{ij} = 1$ iff item $j$ is on strip $i$ (previously $a_{ij} = 1$).}
	\idone{Say that row $i$ covers column $j$ if $x_{ij} = 1$.}
	\idone{Find the smallest number of strips $S \subseteq M$ that contains every item exactly once, i.e. union of strips $= \mathcal{I}$ and intersection $= \emptyset$.}
	\idone{Find minimum cardinality subset $S \subseteq M$ of rows such that each column $j \in N$ is covered by exactly one row $i \in S$.}
\end{itemize}
}

\subsection{Experimental Results - Postoptimisation}
\label{sub:exppostopt}

{\color{myGreen}
\begin{itemize}[leftmargin=*]
	\item Use set $\mathcal{A}$ of feasible packings created during EA in previous section.
	\item C++ circularly linked lists
	\item Designed to not search for all covers, only minimum one, so recursion is restricted to as many levels as the best solution found so far.
	\item Also used Xpress Mosel model (link) to compare.
\end{itemize}
}

%--------------------------------------------------------------------------------------
\section{Conclusion and Further Work}
\label{sec:conclusion}

{\color{myPurple}
\begin{itemize}[leftmargin=*]
	\item Could use selected packings rather than all packings.
	\item Lower bound.
\end{itemize}
}

%--------------------------------------------------------------------------------------
\section{Checklist}
{\color{myAqua}
\begin{itemize}[leftmargin=*]
	\item Vicinal \emph{sum} constraint, not vicinal \emph{score} constraint.
	\item All dashes $'$ not '. 
	\item Citations/references/links.
	\item Abbreviations (SCSPP, SubSCP, AHC etc.) spelling and being used correctly.
	\item Figure and table captions.
	\item References in correct format, title, year, authors, journal etc.
	\item Name of entire process EAX?
	\item Zenodo links DOI.
	\item State specification of computers used for experiments.
	\item Section and subsection titles, capitalisation and spelling.
	\item All figures have same line thickness, dashed line density and thickness, label size, vertex size, and colour (use tikz colours tRed and tBlue).
	\item All figures aligned correctly, subfigures aligned so that the captions are level.
	\item $\backslash$noindent only used when required, after equations, check if needed after definitions/figures/tables etc.
	\item Equations referenced using $\backslash$eqref, not $\backslash$ref.
	\item Tilda $\sim$ before all $\backslash$ref.
	\item Font/colours of figures clear, labels legible.
	\item American/British spelling.
	\item Word repetitions, duplicate statements.
	\item Footnotes.
	\item Compare with previous paper.
	\item Table footnotes, font, spelling.
	\item Pseudocode?
	\item Dominating $\to$ universal.
\end{itemize}		
}

\bibliographystyle{elsarticle-num}
\bibliography{includes/bibliography}

\end{document}