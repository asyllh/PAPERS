\documentclass{elsarticle}

%---- PREAMBLE ----%

% Allows separate files to be used in the main file
\usepackage{subfiles}

% Algorithms
\usepackage{algorithm}
\usepackage[noend]{algpseudocode}
\renewcommand{\algorithmicrequire}{\textbf{Input:}}
\renewcommand{\algorithmicensure}{\textbf{Output:}}
\newcommand{\algorithmicbreak}{\State \textbf{break}}
\newcommand{\Break}{\algorithmicbreak}
\renewcommand{\algorithmicreturn}{\State \textbf{return}}
\newcommand{\algorithmicto}{\textbf{ to }}
\newcommand{\To}{\algorithmicto}
\newcommand{\algorithmicrun}{\State \textbf{call }}
\newcommand{\Run}{\algorithmicrun}
\newcommand{\algorithmicoutput}{\textbf{output }}
\newcommand{\Output}{\algorithmicoutput}
\newcommand{\algorithmictrue}{\textbf{true}}
\newcommand{\True}{\algorithmictrue}
\newcommand{\algorithmicfalse}{\textbf{false}}
\newcommand{\False}{\algorithmicfalse}

% Contains advanced math extensions
\usepackage{amsmath}

% Introduces the *proof* environment and the \theoremstyle command
\usepackage{amsthm}

% Adds new symbols to be used in math mode, e.g. \mathbb
\usepackage{amssymb}

% To declare multiple authors
%\usepackage{authblk}

% Provides extra comands as well as optimisation for producing tables
\usepackage{booktabs}
\newcommand{\ra}[1]{\renewcommand{\arraystretch}{#1}}

% Allows customisation of appearance and placements for figures/tables etc.
\usepackage{caption}

% Adds support for arbitrarily-deep nested lists
\usepackage[inline]{enumitem}

%Support for changing size of font in table footnotes
\usepackage{etoolbox}

% Improves the interface for defining floating objects such as figures/tables
\usepackage{float}

\usepackage{fullpage}

% For easy management of document margins and the document page size
%\usepackage[a4paper]{geometry}

% Allows insertion of graphic files within a document
\usepackage{graphicx}

% Manage links within the document or to any URL when you compile in PDF
\usepackage[colorlinks]{hyperref} 
\usepackage[dvipsnames]{xcolor}
%Tikz colours, used in tikz figures only
\colorlet{tBlue}{RoyalBlue!35!Cerulean} %tikz color
\colorlet{tRed}{Red} %tikz color
\definecolor{tGreen}{HTML}{569909} %tikz color
\definecolor{tOrange}{HTML}{FA7602} %tikz color

%Text colours
\colorlet{myRed}{Red!50!OrangeRed}
\definecolor{myOrange}{HTML}{FA7602}
\definecolor{myGreen}{HTML}{569909}
\definecolor{myAqua}{HTML}{00B1BA} %02BEB8
\definecolor{myBlue}{HTML}{0095FF} %00B3FF
\colorlet{myPurple}{Orchid} 
\colorlet{myPink}{Rhodamine!65!Lavender}
\colorlet{myGray}{Gray!90!White}

% The document class `elsarticle` uses the \AtBeginDocument comman to define the colour of all link types as blue, so we use the \AtBeginDocument command again to overwrite the colour settings to the colors we choose. REMEMBER TO REMOVE THIS BEFORE SUBMITTING.
\AtBeginDocument{% 
	\hypersetup{
		linkcolor=myPurple,
		citecolor=myAqua,
		urlcolor=black
	}
} %end of \AtBeginDocument

\newcommand{\intro}[1]{{\color{myOrange}#1}} % \intro{<text>}, makes <text> orange, use to highlight things to do, urgent, or revisit in introduction section

\newcommand{\ahc}[1]{{\color{myBlue}#1}} % \ahc{<text>}, makes <text> blue, use to highlight things to do, urgent, or revisit in AHC section

\newcommand{\heur}[1]{{\color{myPink}#1}} % \scspp{<text>}, makes <text> pink, use to highlight things to do, urgent, or revisit in SCSPP section

\newcommand{\ea}[1]{{\color{myRed}#1}} % \ea{<text>}, makes <text> red, use to highlight things to do, urgent, or revisit in EA section

\newcommand{\cmsa}[1]{{\color{myGreen}#1}} % \cmsa{<text>}, makes <text> green, use to highlight things to do, urgent, or revisit in CMSA section

\newcommand{\conc}[1]{{\color{myPurple}#1}} % \conc{<text>}, makes <text> purple, use to highlight things to do, urgent, or revisit in Conclusion section

\newcommand{\note}[1]{{\color{myPurple}#1}} % \note{<text>}, makes <text> turquoise, use to highlight things to do, urgent, or revisit in any section


\newcommand{\alert}[1]{{\color{myRed}#1}} % \alert{<text>}, makes <text> red, use to highlight things to do, urgent, or revisit.

\newcommand{\done}[1]{{\color{myGray}#1}} % \done{<text>}, makes <text> gray, use to highlight things that are not required or should be ignored.

\newcommand{\ialert}[1]{{\color{myRed}\item#1}} % \ialert{<text>}, makes <text> a red bullet point, use for bullet points of things to do, urgent, or revisit.

\newcommand{\idone}[1]{{\color{myGray}\item#1}} % \idone{<text>}, makes <text> a gray bullet point, use for bullet points that have been completed, that are not required or should be ignored.

\renewcommand{\idone}[1]{} % hides all \idone bullet points

% Successor of amsmath
\usepackage{mathtools}

\usepackage{multirow}

% No indentation, space between paragraphs
%\usepackage{parskip}

\usepackage{natbib}

%Include standalone .tex files
\usepackage{standalone}

% Define multiple floats (figures/tables) within one environment with individual captions 1a, 1b etc
\usepackage{subcaption}

\usepackage{tabularx}

\usepackage{threeparttable}
\appto\TPTnoteSettings{\footnotesize} % Change font size of table footnotes to footnotesize

\usepackage{tikz}

\usepackage{wrapfig}

\DeclarePairedDelimiter{\floor}{\lfloor}{\rfloor}
\DeclarePairedDelimiter{\ceil}{\lceil}{\rceil}

%Theorem style
\theoremstyle{plain}% default
\newtheorem{theorem}{Theorem}
%\newtheorem{corollary}{Corollary}
\newtheorem{definition}{Definition}

%\theoremstyle{definition}
%\newtheorem{definition}{Definition}
%\newtheorem{proposition}{Proposition}
%\newtheorem{exmp}{Example}[section]

\begin{document}
	
\begin{frontmatter}
\title{An Evolutionary Algorithm for the Score-Constrained Strip-Packing Problem}
\author{Asyl L. Hawa}
\author{Rhyd Lewis}
\author{Jonathan M. Thompson}
\address{School of Mathematics, Cardiff University, Senghennydd Road, Cardiff, UK}
\end{frontmatter}

%--------------------------------------------------------------------

\section{Introduction}

A set $\mathcal{I}$ of $n$ rectangular cardboard items of equal height $H$ and varying widths $w_i \in \mathbb{Z}^+$ are to be packed on a strip from left to right. Each item $i \in \mathcal{I}$ is marked with two vertical lines, called score lines, in predetermined places. The distance between each score line and the edge of the item are score widths, $a_i, b_i \in \mathbb{Z}^+$. Once the items are packed onto the strip, a pair of knives on a bar simultaneously cut along the score lines of two adjacent items to aid the folding process. Due to the manner in which the knives are secured onto the bar, the knives must maintain a set distance from one another, a so-called ``minimum scoring distance''. Therefore, in order for the knives to score all of the items in the correct locations, the distance between two score lines of adjacent items must exceed the minimum scoring distance. Hence, the problem involves packing the items onto the strip in such a way that the sum of all adjacent score widths exceeds the minimum scoring distance, as stated in the following definition:

\begin{definition}
	\label{defn:subprob}
	Let $\mathcal{I}$ be a set of items with widths $w_i \in \mathbb{Z}^+$, score widths $a_i, b_i \in \mathbb{Z}^+$, and equal height $H$. Given a minimum scoring distance $\tau$, the Score-Constrained Packing Sub-Problem (SCPSP) involves finding an arrangement of the items in $\mathcal{I}$ such that the sum of every pair of adjacent score widths is greater than or equal to $\tau$.
\end{definition}



\begin{itemize}
	\item Define COP, single strip problem and multi-strip problem (SCSPP)
	\item Other examples of where COP is used
	\item Cite Goulimis, previous Heuristics paper, follow on
	\item Lit review, Lewis TSP method
	\item BPP special case of SCSPP where $\tau = 0$, hence SCSPP NP-hard
	\item Need to decide which items to pack on which strips and how the items should be packed, different orientations, s.t. minimum scoring distance met
	\item Show that sub-problem can be seen as an instance of COP, minimum scoring distance = vicinal sum constraint
	\item Describe rest of paper, AHC single strip for COP, Evolutionary Algorithm and Postopt
	\item Two phase approach, cite vertex colouring paper
	\item Minimum scoring distance $\tau$ approx 70mm in industry
	\item $\frac{n!}{2} 2^n$ distinct arrangement, combinatorial explosion
	\item COP infinite width, strips given fixed width
	\item Outer score widths ignored
\end{itemize}

\begin{definition}
	Let $\mathcal{M}$ be a multiset of unordered pairs, $\mathcal{M} = \{\{a_1, b_i\}, \{a_2, b_2\},..., \{a_n, b_n\}\}$, $a_i, b_i \in \mathbb{Z}^+$, and let $\mathcal{T}$ be an ordering of the elements of $\mathcal{M}$ such that each element is a tuple. Given a fixed value $\tau \in \mathbb{Z}^+$, the Constrained Ordering Problem (COP) consists of finding a solution $\mathcal{T}$ such that
	\begin{equation}
		\label{eqn:vsc}
		\emph{\textbf{rhs}}(i) + \emph{\textbf{lhs}}(i+1) \geq \tau \hspace{5mm} \forall \hspace{1mm} i \in \{1,2,..., n-1\},
	\end{equation}
	where \emph{\textbf{lhs}($i$)} and \emph{\textbf{rhs}($i$)} denote the left- and right-hand values of the $i$th tuple. The inequality is referred to as the \emph{vicinal sum constraint}.
\end{definition}

For example, given the COP instance $\mathcal{M} = \{\{1,2\}, \{1,6\}, \{2,3\}, \{2,4\}, \{3,5\}, \{4,5\}\}$, we have a solution $\mathcal{T} = \langle(1,6), (2,4), (3,5), (2,3), (4,5), (2,1)\rangle$

\begin{definition}
	Let $\mathcal{I}$ be a set of $n$ rectangular items of height $H$ with varying widths $w_i \in \mathbb{Z}^+$ and score widths $a_i, b_i \in \mathbb{Z}^+$ for each item $i \in \mathcal{I}$. Given a minimum scoring distance $\tau \in \mathbb{Z}^+$, the Score-Constrained Strip-Packing Problem (SCSPP) consists of finding the minimum number of strips of height $H$ and width $W$ required to pack all items in $\mathcal{I}$ such that the sum of every pair of adjacent score widths is greater than or equal to $\tau$ and no strip is overfilled.
\end{definition}

\begin{figure}	
	\centering
	\begin{subfigure}[h]{0.45\textwidth}
		\includestandalone[width=\textwidth]{figures/items}
		\label{fig:items}
	\end{subfigure} 
	\begin{subfigure}[h]{0.75\textwidth}
		\includestandalone[width=\textwidth]{figures/itemsaligned}
		\label{fig:itemsaligned}
	\end{subfigure}
	\caption{Example of a single strip-packing problem and a corresponding solution with $\tau = 7$.}
	\label{fig:items/aligned}
\end{figure}

In the next section, we will provide a brief overview of the polynomial-time algorithm used to solve the COP. Section 3 will explain the difficulties associated with the SCSPP, and analyse bespoke heuristics. An evolutionary algorithm for the SCSPP is detailed in Section 4, along with results from rigorous experiments.  

\section{AHC}

From the instance $\mathcal{M}$ of the COP, we produce a graph $G$ consisting of $2n$ vertices, one for each element of $\mathcal{M}$, in non-increasing order. An extra pair of dominating vertices are added to $G$, which have values equal to $\tau$. $G$ comprises two edge sets: $B$, which contains edges between vertices that are partners (that is, whose values make up a pair in $\mathcal{M}$); and $R$, which contains edges between vertices whose values add up to greater than or equal to $\tau$, and that are not partners.

\textcolor{OrangeRed}{THRESHOLD GRAPH FIGURE}

A Hamiltonian cycle on a graph $G$ is a cycle that visits every vertex of $G$ exactly once. From this, we define a specific type of Hamiltonian cycle:

\begin{definition}
	\label{defn:althamcycle}
	Let $G = (V, B \cup R)$ be a simple, undirected graph where each edge is a member of one of two sets, $B$ or $R$. $G$ contains an alternating Hamiltonian cycle if there exists a Hamiltonian cycle such that successive edges alternate between sets $B$ and $R$.
\end{definition}

Due to the manner in which $G$ is constructed, it can be seen that the edges in $R$ represents all possible pairings of values from different pairs that meet the vicinal sum constraint. All pairs of elements must be present in the final solution, and so the edges in $B$ cannot be altered or removed. Clearly, an alternating Hamiltonian cycle on $G$ corresponds to a feasible ordering $\mathcal{T}$. Therefore, our task involves finding a matching $R' \subseteq R$ that, together with the edges in $B$, forms an alternating Hamiltonian cycle on $G$.

The Maximum Cardinality Matching (MCM) algorithm is used to produce a matching $R'$ from $R$. MCM takes each vertex $v_1, v_2,...,v_{2n+2}$ and adds to $R'$ the edge from $R$ connecting $v_i$ to the highest-indexed vertex $v_j$ that is not incident to an edge in $R'$. Pairs of vertices incident to edges in $R'$ are said to be ``matched''.

BR searches for an edge whose lower-indexed vertex is adjacent to the next edge's higher-indexed vertex, and that is not in the same component of $G'$ as the next edge. This edge starts the set $R_1$, and succeeding edges are added to $R_1$ provided the conditions hold and the next edge is not in the same component as any of the edges in $R_1$. BR continues through the list creating these sets until the penultimate edge has been reached. 





\begin{itemize}
	\item Preliminary check
	\item Model instance graphically, graph $G$ has $2n+2$ vertices, including 2 extra dominating vertices (which will be removed at the end) and two edge sets, $B$ and $R$
	\item $B$ = set of edges between vertices that are partners (that make up a pair in the instance $\mathcal{M}$), ``blue'' edges. $R$ - set of edges between vertices that are not partners and whose values add up to $\geq \tau$
	\item Define Alt ham cycle.
	\item Aim - to find set of edges $R' \subseteq R$ that forms an alternating hamiltonian cycle with the edges in $B$
	\item Alt ham cycle NP-hard as it generalises Ham cycle propostion (proof), but structure of these graphs means can solve in polynomial time
	\item MCM - match each vertex with largest possible vertex, create set $R'$. If $|R'| < n+1$, no solution exists, not enough edges to form cycle, end. Else if $|R'| = n+1$, i.e. matching, then $G' = (V, B \cup R')$ is 2-reg graph, consists of cycles $C_1,...,C_l$.
	\item If $l = 1$, then $G'$ is alt ham cycle, solution found, end (remove dom vertices). Else we need to find edges from $R\backslash R'$ that can act as bridges between the components and join them together into a single cycle. 
	\item Use BR to find these edges. List edges in order, go through to find edges that meet specific conditions. Continue until penultimate edge reached. BR produces sets $R_1, R_2,...$
	\item If no sets produced, no solution exists, end. Else if $\exists$ set such that $|R_i| = l$, use connecting procedure to join components together, solution found, end.
	\item Else multiple sets need to be used, run MBR to find two or more sets that overlap correctly. If collection $\mathcal{R}^*$ found, connecting procedure on all sets in collection, solution found, end. Else, no solution exists, end.
	\item Guaranteed to find solution, if one exists, in $O(n^2)$ time.
\end{itemize}

\begin{itemize}
	\item Condensed version of algorithm from previous paper
	\item Theorem/Proof polynomial time $O(n^2)$
	\item Hamiltonian cycle characteristics, NP-hard, Karp 21 NP-Complete problem (both direct and undirected), complexity
	\item Pseudocode/brief overview of AHC algorithm in steps
	\item Preliminary checks - these were not included in previous paper, speed
	\item MBR types, check that Beckers method is redundant
	\item Add dominating vertices to help produce Ham cycle, remove at end, outer score widths ignored
	\item Theorem MCM guarantee $O(n \lg n)$
\end{itemize}

\begin{figure}[H]	
	\centering
	\begin{subfigure}[h]{0.4\textwidth}
		\includestandalone[width=\textwidth]{figures/matching}
	\end{subfigure} \quad
	\begin{subfigure}[h]{0.4\textwidth}
		\includestandalone[width=\textwidth]{figures/mps}
	\end{subfigure}
	\caption{Subgraph $G'= (V, B \cup R')$ created after MCM. When in planar form, it is clear that $l = 2$.}
	\label{fig:mps}
\end{figure}

\section{Heuristics for the SCSPP}

For an instance of the SCSPP, a feasible solution is represented by the set $\mathcal{S} = \{S_1, S_2, ..., S_k\}$ such that
\begin{subequations}
	\begin{align}
	\bigcup\nolimits_{i=1}^{|\mathcal{S}|} S_i &= \mathcal{I}, \label{eqn:packall}\\[3pt]
	S_i \cap S_j &= \emptyset \hspace{5mm} \forall \hspace{1mm} i, j \in \{1, 2, ..., |\mathcal{S}|\}, \hspace{2mm} i \neq j, \label{eqn:nooverlap} \\[3pt]
	\sum\nolimits_{i=1}^{|S_j|}w_i & \leq W \hspace{3mm} \forall \hspace{1mm} S_j \in \mathcal{S}, \label{eqn:capacity} \\[3pt]
	\textup{\textbf{rhs}}(i) + \textup{\textbf{lhs}}(i+1) &\geq \tau \hspace {5mm} \forall \hspace{1mm} i \in \{1, 2, ..., |S_j|-1\}, \hspace{2mm} \forall \hspace{1mm} S_j \in \mathcal{S}. \label{eqn:vscstrip}
	\end{align}
\end{subequations}

\begin{itemize}
	\item SCSPP at least as hard as BPP, NP-hard
	\item Why heuristics?
	\item lowerbound, optimal solution, theoretical minimum
	\item theoretical min less accurate
	\item State advantages and disadvantages of MFFD and MFFD$^+$
	\item Differences between SCSPP and BPP
\end{itemize}

\subsection{Heuristics}
\begin{itemize}
	\item For both heuristics, if current item is placed on a new strip, it is placed in a regular orientation - smaller score width on the left-most side of the strip.
	\item MFFD - operates in the same way as FFD, except additional step of checking that VSC is met with one of the score widths of the item and the right-most score width on the strip, i.e. check that $S \cup \{i\} \in \mathcal{F}$.
	\item MFFD$^+$ - if item fits on strip, run AHC on items on strip and current item. If solution found, replace current arrangement of items on strip with arrangement found by AHC including new item, else move on to next strip.
\end{itemize}


\subsection{Experimental Results - Heuristics}
\begin{itemize}
	\item Time it takes to run heuristics
	\item Tables of comparisons
	\item State that there are 20 types on average for real instances (no longer in table)
\end{itemize}

For our experiments, we produced two different types of instances: ``artificial'' instances, which contain items of varying widths and score widths; and ``real'' instances, which have multiple items of the same dimension. For each type 1000 instances were generated using sets of 100 and 500 items, giving a total of 4000 problem instances. For all instances, the minimum scoring distance $\tau$ was set to 70mm - the industry standard. All items have widths $w_i \in [150,1000]$ and score widths $a_i, b_i \in [1,70]$ selected uniform randomly, and equal height $H=1$. To compare our heuristics, we use two different strips widths, 2500 and 5000.

\begin{table}[h!]
	\centering
\caption{MFFD vs MFFD$^+$}
\begin{threeparttable}
\begin{tabular}{cccccccccccc}\toprule
	& & & \multicolumn{4}{c}{MFFD} &\phantom{a}& \multicolumn{4}{c}{MFFD$^+$}\\
	\cmidrule{4-7} \cmidrule{9-12}
	Instance & $W$ & $t$ & $|\mathcal{S}|$\tnote{1} & $\# t$ & $q$ & $f(\mathcal{S})$ && $|\mathcal{S}|$ & $\# t$ & $q$ & $f(\mathcal{S})$\\ \midrule	
	a,100 & 2500 & 23.323 & 30.754 & 0 & 1.320 & 0.686 && 28.457 & 26 & 1.221 & 0.771 \\
	a,100 & 5000 & 11.922 & 23.583 & 0 & 1.982 & 0.412 && 19.881 & 7 & 1.670 & 0.543  \\
	\midrule
	a,500 & 2500 & 114.942 & 140.206 & 0 & 1.220 & 0.781 && 132.647 & 0 & 1.154 & 0.842 \\
	a,500 & 5000 & 57.722 & 103.209 & 0 & 1.789 & 0.499 && 89.544 & 0 & 1.552 & 0.609 \\
	\midrule
	\midrule
	r,100 & 2500 & 23.473 & 37.069 & 5 & 1.600 & 0.549 && 35.419 & 16 & 1.523 & 0.597 \\
	r,100 & 5000 & 11.981 & 32.348 & 1 & 2.731 & 0.288 && 29.611 & 5 & 2.497 & 0.347 \\
	\midrule
	r,500 & 2500 & 115.239 & 184.106 & 0 & 1.612 & 0.552 && 177.249 & 0 & 1.551 & 0.593 \\
	r,500 & 5000 & 57.865 & 163.819 & 0 & 2.860 & 0.279 && 153.416 & 0 & 2.678 & 0.322 \\
	\bottomrule
\end{tabular}
\begin{tablenotes}
	\item[1] Mean from 1000 instances
\end{tablenotes}	
\end{threeparttable}	
\label{table:MFFD}
\end{table}

\begin{table}[h!]
\centering
\caption{average number of items per strip - MFFD vs MFFD$^+$}
	\begin{tabular}{ccccc}\toprule
		Instance & $W$ & items/lb & MFFD & MFFD$^+$ \\ \midrule	
		a,100 & 2500 & 4.288 & 3.252 & 3.514 \\
		a,100 & 5000 & 8.388 & 4.240 & 5.030 \\
		\midrule
		a,500 & 2500 & 4.350 & 3.566 & 3.769 \\
		a,500 & 5000 & 8.662 & 4.845 & 5.584 \\
		\midrule
		\midrule
		r,100 & 2500 & 4.260 & 2.698 & 2.823 \\
		r,100 & 5000 & 8.347 & 3.091 & 3.377 \\
		\midrule
		r,500 & 2500 & 4.339 & 2.716 & 2.821 \\
		r,500 & 5000 & 8.641 & 3.052 & 3.259 \\
		\bottomrule
	\end{tabular}	
\end{table}


\begin{itemize}
	\item Although MFFD$^+$ better than MFFD, still disadvantages
	\item Initially all items in decreasing order, may be a better ordering
	\item only packing one item at a time, one item may not be feasible on a strip but if two were packed at the same time could be feasible
	\item explain benefit of using Evolutionary algorithm, lead on to next section
\end{itemize}

\section{Evolutionary Algorithm}
Throughout the EA we maintain two sets, $\mathcal{A}$ and $\mathcal{B}$, containing subsets of items that AHC has determined produce feasible or infeasible orderings, respectively. When an instance of the subproblem occurs, these sets are searched before calling AHC; hence AHC is executed at most once for each distinct subset of items.

\begin{itemize}
	\item Only MFFD$^+$ used in EA
	\item Create initial population, one solution using MFFD$^+$, the rest using MFFR$^+$
	\item Each solution in population is also mutated and local search is applied
	\item Choose two parent solutions from population
	\item Apply one of three recombination operators
	\item Paragraph for each operator, GGA, GPX', and GPN
	\item Explain why GPX cannot be used, cite Lewis paper
	\item Describe repair procedure using MFFD$^+$
	\item Section explaining Local Search, PairPair, PairSin, SinSin, MoveSin
	\item Preliminary check before calling AHC
	\item Once offspring is produced, calculate fitness of parents and offspring, replace worst parent with offspring in population
	\item Fitness $f(\mathcal{S}) = \frac{\sum_{S \in \mathcal{S}} (A(S)/W)^2}{|\mathcal{S}|}$ Falkenaur 1998
\end{itemize}

\subsection{Mutation}
Take candidate solution, permute strips, select a random number of strips from the solution to insert into partial solution $\mathcal{S}_X$, and the rest into $\mathcal{S}_Y$. Apply local search on $\mathcal{S}_X$ and $\mathcal{S}_Y$ to produce a single feasible solution.

\subsection{Crossover Operator}
\begin{itemize}
	\item May have duplicate or missing items
	\item Note the differences between BPP and SCSPP, cannot simply remove individual items from strips, may cause the arrangement of remaining items to become infeasible (VSC not met between items)- hence why GPX cannot be used (Lewis and Holborn paper)
	\item GGA $1 \leq i < j \leq |\mathcal{S}_Y|$ \textbf{and} cannot have $i = 1 \land j = |\mathcal{S}_Y|$ unlike in Rhyd paper - ensures that at least 2 strips from $\mathcal{S}_Y$ are in offspring, and prevents all strips from $\mathcal{S}_Y$ from being selected, as this would mean that all strips from $\mathcal{S}_X$ would be deleted from the offspring, and so offspring will not have strips from both parents. Ensures that offspring has strips from both parents.
	\item If missing items, repair procedure, then local search
	\item GGA - Permute strips of second parent solution, then choose two strips randomly, add all strips between two chosen strips (including chosen strips) and add to offspring. Then from first parent, only add strips to offspring containing items that are not in offspring. Repair procedure on missing items, if any.
	\item GPX' - Alternate between parent solutions, add fullest strip to offspring, delete strips from other parent containing items in offspring. Continue until $min(|\mathcal{S}_X|,|\mathcal{S}_Y|)-1$ strips in offspring. Repair procedure on missing items, if any. Initially, if both parents have fullest strip, break ties randomly.
	\item GPN - Operates in the same manner as GPX', except chose strip containing the most items. Initially, if both parents have strip containing same number of max items, choose the parent whose strip is the fullest. If equal for both strips, choose randomly.
	\item Once offspring produced, mutate and apply local search before replacing worst parent.
\end{itemize}

(GGA) The strips of the second parent solution $\mathcal{S}_2$ are permuted, and two strips in $\mathcal{S}_2$ are selected randomly. These strips, along with all strips in between, are inserted into the offspring solution $\mathcal{S}$. The operator then adds to the offspring all strips from $\mathcal{S}_1$ that do not contain items already present in the offspring. Any items that are not in $\mathcal{S}$ are used to created a feasible partial solution $\mathcal{S}'$ using MFFD$^+$. Local search is then used on $\mathcal{S}$ and $\mathcal{S}'$ to produce a full offspring solution.


\subsection{Local Search}
\begin{itemize}
	\item Takes in two feasible partial solutions, produces one feasible solution
	\item Initially permute the strips of both solutions
	\item Attempt to swap pairs of items, a pair of items with a single item, swap single items, or move an item, repeat until no changes
	\item Apply MFFD$^+$ on any items not packed
	\item When swapping/moving items, AHC is used on the sub-problems
	\item Also keep track of feasible/infeasible arrangements, check these sets first before running AHC
	\item Therefore for any collection of items AHC is only called once (or not at all if preliminary checks failed)
	\item Unlike Lewis/Falkenauer, order matters, so need to use AHC
\end{itemize}

\subsection{Experimental Results - EA}
\begin{itemize}
	\item Types of instances - artificial, real
	\item Problem instance generator will produce 1000 instances for four different sets, number of items = 100 and 500 and artificial and real instances, 4000 problem instances in total
	\item Each of these sets will then be run 6 times using different arguments - strip width = 2500 and 5000, and either GGA, GPX', or GPN
	\item Overall there will be 24 types of outputs from EA
	\item In EA output file to compare:
	\begin{itemize}
		\item Proportion of times feasPacking/infeasPacking sets used instead of AHC
		\item Number of strips in best solution found
		\item Number of iterations of EA within the given time limit
		\item Fitness value of the best solution found
		\item If best solution found is equivalent to the lowerbound
	\end{itemize}
	\item Keep track of best fitness/best solution/time at each iteration, if better solution found then write to file, produce graph
	\item At the end of each EA iteration, add all strips from offspring to feasPacking set to use in post optimisation phase
	\item number of solution in initial population = 25
	\item Number of items - 100, 500
	\item Time limit for EA - 5 minutes
	\item Width of strips - 2500, 5000
	\item Recombination operators - GGA, GPX', GPN
\end{itemize}

\begin{table}[h!]
	\centering
	\caption{xOver comparisons}
	\begin{tabular}{cccccccccccc}\toprule
		& & & &\multicolumn{2}{c}{GGA} &\phantom{a}& \multicolumn{2}{c}{GPX'} &\phantom{a}& \multicolumn{2}{c}{GPN}\\
		\cmidrule{5-6} \cmidrule{8-9} \cmidrule{11-12}
		Instance & $W$ & $t$ && $|\mathcal{S}|$ & $\# t$ && $|\mathcal{S}|$ & $\# t$ && $|\mathcal{S}|$ & $\# t$ \\ \midrule	
		a,100 & 2500 & 23.323 && 23.483 & 931 && 23.357 & 977 && 23.372 & 966 \\
		a,100 & 5000 & 11.922 && EA151 & - && EA152 & - && EA153 & -\\
		\midrule
		a,500 & 2500 & 114.942 && 116.681 & 264 && 117.041 & 213 && 116.604 & 277 \\
		a,500 & 5000 & 57.722 && EA551 & - && EA552 & - && EA553 & -\\
		\midrule
		\midrule
		r,100 & 2500 & 23.473 && ER121 & - && ER122 & - && ER123 & -\\
		r,100 & 5000 & 11.981 && ER151 & - && ER152 & - && ER153 & -\\
		\midrule
		r,500 & 2500 & 115.239 && ER521 & - && ER522 & - && ER523 & -\\
		r,500 & 5000 & 57.865 && ER551 & - && ER552 & - && ER553 & -\\
		\bottomrule
	\end{tabular}	
	\label{table:EA}
\end{table}

\section{Postoptimisation}
\begin{itemize}
	\item Exact cover formulation, NP-hard, state the IP, describe DLX
	\item cite Knuth dancing steps
	\item Use strips in feasPacking for post opt
	\item Compare with EA output - is a better solution found, or a solution with the same number of strips but a better fitness value?
	\item Is the post opt phase able to find a solution equal to the lowerbound?
	\item Post opt will only ever find a solution equal to or better than the solution found in EA, never worse
	\item Execution time
	\item Set cover problem is optimisation problem, find min number of sets
	\item exact cover problem is decision problem, does a set exist
	\item however since we add every item on its own strip to feasPacking, we know that a set exists
	\item problem is to find minimum number of strips that covers all items and contains every item exactly once
	\item $A = (a_{ij})$ - $m$ x $n$ matrix
	\item $M = \{1, 2,..., m\}$ - rows of the matrix, each row $i \in M$ is a strip
	\item $N = \{1, 2,...,n\}$ - columns of the matrix, each column $j \in N$ is an item
	\item $(a_{ij}) = 1$ iff item $j$ is on strip $i$
	\item Say that row $i$ covers column $j$ if $a_{ij} = 1$
	\item Find the smallest number of strips $S \subseteq M$ that contains every item exactly once, i.e. union of strips $= \mathcal{I}$ and intersection $= \emptyset$
	\item Find minimum cardinality subset $S \subseteq M$ of rows such that each column $j \in N$ is covered by exactly one row $i \in S$
	\item Xpress mosel model
\end{itemize}
\[x_i =
\begin{cases} 
1 & \text{if } i \in S \\
0 & \text{otherwise} 
\end{cases}
\]

\begin{align*}
\text{minimise} &\sum_{i \in M} x_i \\
\text{subject to} &\sum_{i \in M} a_{ij} x_i = 1 \hspace{5mm} \forall j \in N \\
&x_i \in \{0,1\} \hspace{5mm} \forall i \in M
\end{align*}

\subsection{Experimental Results - Postoptimisation}
\begin{itemize}
	\item
\end{itemize}

\section{Conclusion}
\begin{itemize}
	\item Could use selected packings rather than all packings
\end{itemize}

\cite{becker2010}, \cite{becker2015}, \cite{coffman1978}, \cite{coffman1984}, \cite{dosa2007} \cite{eilon1971} \cite{falkenauer1992} \cite{falkenauer1996}, \cite{garey1972}, \cite{garey1979}, \cite{garraffa2016}, \cite{gilmore1961}, \cite{gilmore1963}, \cite{goulimis2004}, \cite{haggkvist1977}, \cite{hawa2018}, \cite{hung1978}, \cite{johnson1973}, \cite{johnson1974fast}, \cite{johnson1974worst}, \cite{karp1972}, \cite{knuth2000}, \cite{korf2002}, \cite{levine2004}, \cite{lewis2009}, \cite{lewis2011}, \cite{lewis2017}, \cite{mahadev1994}, \cite{mahadev1995}, \cite{malaguti2008}, \cite{martello1990a}, \cite{martello1990b}, \cite{quiroz2015}

\bibliographystyle{elsarticle-num}
\bibliography{includes/bibliography}

\end{document}