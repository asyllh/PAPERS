\documentclass[a4paper,11pt]{article}

% Algorithms
\usepackage{algorithm}
\usepackage[noend]{algpseudocode}
\renewcommand{\algorithmicrequire}{\textbf{Input:}}
\renewcommand{\algorithmicensure}{\textbf{Output:}}
\newcommand{\algorithmicbreak}{\State \textbf{break}}
\newcommand{\Break}{\algorithmicbreak}
\renewcommand{\algorithmicreturn}{\State \textbf{return}}
\newcommand{\algorithmicto}{\textbf{ to }}
\newcommand{\To}{\algorithmicto}
\newcommand{\algorithmicrun}{\State \textbf{call }}
\newcommand{\Run}{\algorithmicrun}
\newcommand{\algorithmicoutput}{\textbf{output }}
\newcommand{\Output}{\algorithmicoutput}
\newcommand{\algorithmictrue}{\textbf{true}}
\newcommand{\True}{\algorithmictrue}
\newcommand{\algorithmicfalse}{\textbf{false}}
\newcommand{\False}{\algorithmicfalse}
\newcommand{\algorithmicand}{\textbf{ and }}
\newcommand{\AAnd}{\algorithmicand}
\newcommand{\algorithmicnot}{\textbf{not}}
\newcommand{\Not}{\algorithmicnot}
\newcommand{\algorithmicor}{\textbf{ or }}
\newcommand{\Or}{\algorithmicor}

\usepackage{amsmath}
\usepackage{amsthm}
\usepackage{amssymb}
\usepackage{booktabs}
\newcommand{\ra}[1]{\renewcommand{\arraystretch}{#1}}
\usepackage{caption}
\usepackage{comment}
\usepackage[inline]{enumitem}
\usepackage{float}
\usepackage[a4paper]{geometry}
\usepackage{graphicx}
\usepackage[colorlinks]{hyperref} 
\usepackage[dvipsnames]{xcolor}
%Tikz colours, used in tikz figures only
\colorlet{tBlue}{RoyalBlue!35!Cerulean} %tikz color
\colorlet{tRed}{Red} %tikz color
\definecolor{tGreen}{HTML}{569909} %tikz color
\definecolor{tOrange}{HTML}{FA7602} %tikz color
\definecolor{tLightGreen1}{HTML}{C1E685} %tikz color
\definecolor{tLightOrange1}{HTML}{FFCD4F} %tikz color
\colorlet{tLightGreen}{LimeGreen!70!OliveGreen!45!White}
\colorlet{tLightOrange}{Dandelion!65!White}
\definecolor{tLightPink}{HTML}{FFD4EB} %tikz color
\definecolor{tLightBlue}{HTML}{CEF0FF} %tikz color
%Text colours
\colorlet{myRed}{Red!50!OrangeRed}
\newcommand{\rev}[1]{{\color{myRed}#1}}
\usepackage{mathtools}
\usepackage{multirow}
\usepackage{soul}
\usepackage{standalone}
\usepackage{subcaption}
\usepackage{tabularx}
\usepackage{threeparttable}
\usepackage{tikz}
\usetikzlibrary{shapes.geometric}
\usepackage{wrapfig}
\DeclarePairedDelimiter{\floor}{\lfloor}{\rfloor}
\DeclarePairedDelimiter{\ceil}{\lceil}{\rceil}
\newtheorem{theorem}{Theorem}
\newtheorem{definition}{Definition}

%-------------------------------------------------

\begin{document}
\title{Appendix\vspace{-5mm}}
\date{}
\maketitle
\section{The Bridge-Cover Recognition Algorithm}

\begin{algorithm}[H]
\caption{BCR ($G'=(V, B \cup R')$)}
\begin{algorithmic}[1]
	\State create list $\mathcal{L}$ of edges in $R'$
	\State let $v_k$ be the lower-indexed vertex of the $k$th edge in $\mathcal{L}$, and let $m(v_k)$ be the higher-indexed vertex of the $k$th edge in $\mathcal{L}$
	\State $\mathcal{R}'' \gets \emptyset$
	\State $\mathcal{C}[j] \gets 0$ $\forall j = 1,\dotsc,z$ \Comment Records the components that have been covered
	\State $n\gets 0$ \Comment Number of components that have been covered
	\State $i \gets 1$ \Comment Index of subsets $R''_i$
	\While{$k < |\mathcal{L}|-2 \AAnd (\{v_k, m(v_{k+1})\} \notin R \Or (\{v_k,m(v_k)\} \in C_p \newline \hspace*{4mm} \AAnd \{v_{k+1},m(v_{k+1})\} \in C_p))$}
		\State $k \gets k+1$
	\EndWhile
	\If{$\{v_k,m(v_{k+1})\} \in R \AAnd \{v_k,m(v_k)\} \in C_p \AAnd \{v_{k+1},m(v_{k+1})\} \in C_q$}
		\State $R''_i \gets \{\{v_k,m(v_{k})\}\} \cup \{\{v_{k+1},m(v_{k+1})\}\}$
		\State $\mathcal{C}[p] \gets 1$, $\mathcal{C}[q] \gets 1$
		\State $n \gets n+2$
		\State $k \gets k+1$
		\While{$k < |\mathcal{L}|-1 \AAnd \{v_k,m(v_{k+1})\} \in R \newline \hspace*{10.2mm}\AAnd \{v_{k+1},m(v_{k+1})\} \in C_r : \mathcal{C}[r] = 0$}
			\State $k \gets k+1$
			\State $R''_i \gets R''_i \cup \{\{v_{k},m(v_{k})\}\}$
			\State $\mathcal{C}[r] \gets 1$
			\State $n \gets n+1$
		\EndWhile
		\State $\mathcal{R}'' \gets R''_i$
	\EndIf
	\If{$\mathcal{R}'' = \emptyset$}
		\State infeasible, \textbf{end}
	\ElsIf{$n = z$}
		\State feasible, \textbf{go to} Line 93
	\EndIf
	\State remove edges that are in $\mathcal{R}''$ from $\mathcal{L}$
	\State added $\gets \False$
	\State $\overline{\mathcal{C}} \gets \mathcal{C}$
	\State $\overline{n} \gets n$
\algstore{bcr}
\end{algorithmic}
\end{algorithm}
\begin{algorithm}[H]
\begin{algorithmic}[1]		
\algrestore{bcr}
	\While{$\Not$ added}
		\State $k \gets 1$			
		\Repeat
			\State $\overline{R''} \gets \emptyset$
			\While{$k < |\mathcal{L}|-2 \AAnd (\{v_k, m(v_{k+1})\} \notin R \Or (\{v_k,m(v_k)\} \in C_p \newline \hspace*{16mm}\AAnd \{v_{k+1},m(v_{k+1})\} \in C_p))$}		
				\State $k \gets k+1$
			\EndWhile			
			\If{$\{v_k,m(v_{k+1})\} \in R \AAnd \{v_k,m(v_k)\} \in C_p \AAnd \{v_{k+1},m(v_{k+1})\} \in C_q \newline \hspace*{16mm} \AAnd ((\mathcal{C}[p] = 1 \AAnd \mathcal{C}[q] = 0) \Or (\mathcal{C}[p] = 0 \AAnd \mathcal{C}[q] = 1))$}
				\State $R''_i \gets \{\{v_k,m(v_{k})\}\} \cup \{\{v_{k+1},m(v_{k+1})\}\}$
				\State $\mathcal{C}[p] \gets 1$, $\mathcal{C}[q] \gets 1$
				\State $n \gets n+1$
				\If{$n < z$}
					\State $k \gets k+1$
					\While{$k < |\mathcal{L}|-1 \AAnd \{v_k,m(v_{k+1})\} \in R \newline \hspace*{27mm}\AAnd \{v_{k+1},m(v_{k+1})\} \in C_r : \mathcal{C}[r] = 0$}
						\State $k \gets k+1$
						\State $R''_i \gets R''_i \cup \{\{v_{k},m(v_{k})\}\}$
						\State $\mathcal{C}[r] \gets 1$
						\State $n \gets n+1$
					\EndWhile 
				\EndIf %End if not all cycles covered
				\State $\mathcal{R}'' \gets \mathcal{R}'' \cup R''_i$
				\State added $\gets \True$
				\State $\overline{\mathcal{C}} \gets \mathcal{C}$, $\overline{n} \gets n$
				\State $i \gets i+1$
			\ElsIf{$\{v_k,m(v_{k+1})\} \in R \AAnd \{v_k,m(v_k)\} \in C_p \AAnd \newline \hspace*{17mm} \{v_{k+1},m(v_{k+1})\} \in C_q \AAnd ((\mathcal{C}[p] = 0 \AAnd \mathcal{C}[q] = 0))$}	
				\State overlap $\gets \False$
				\State $\overline{R''} \gets \{\{v_k,m(v_{k})\}\} \cup \{\{v_{k+1},m(v_{k+1})\}\}$
				\State $\overline{\mathcal{C}}[p] \gets 1$, $\overline{\mathcal{C}}[q] \gets 1$
				\State $\overline{n} \gets \overline{n}+2$
				\State $k \gets k+1$
				\While{$k < |\mathcal{L}|-1 \AAnd \{v_k,m(v_{k+1})\} \in R \newline \hspace*{21.3mm}\AAnd \{v_{k+1},m(v_{k+1})\} \in C_r : \overline{\mathcal{C}}[r] = 0$}
					\State $k \gets k+1$
					\State $\overline{R''} \gets \overline{R''} \cup \{\{v_{k},m(v_{k})\}\}$
					\State $\overline{\mathcal{C}}[r] \gets 1$
					\State $\overline{n} \gets \overline{n}+1$
				\EndWhile
				\If{$k < |\mathcal{L}|-1 \AAnd \{v_k,m(v_{k+1})\} \in R \newline \hspace*{21.3mm}\AAnd \{v_{k+1},m(v_{k+1})\} \in C_r : \overline{\mathcal{C}}[r] = 1$}
					\If{$\mathcal{C}[r] = 1$}
						\State overlap $\gets \True$
						\State $k \gets k+1$
						\State $\overline{R''} \gets \overline{R''} \cup \{\{v_{k},m(v_{k})\}\}$
\algstore{bcr2}				
\end{algorithmic}
\end{algorithm}
\begin{algorithm}[H]
\begin{algorithmic}[1]		
\algrestore{bcr2}								
						\If{$n < z$}
							\State $k \gets k+1$
							\While{$k < |\mathcal{L}|-1 \AAnd \{v_k,m(v_{k+1})\} \in R \newline \hspace*{38.7mm}\AAnd \{v_{k+1},m(v_{k+1})\} \in C_r : \overline{\mathcal{C}}[r] = 0$}
								\State $k \gets k+1$
								\State $\overline{R''} \gets \overline{R''} \cup \{\{v_{k},m(v_{k})\}\}$
								\State $\overline{\mathcal{C}}[r] \gets 1$
								\State $\overline{n} \gets \overline{n}+1$
							\EndWhile
						\EndIf
					\ElsIf{$\mathcal{C}[r] = 0$}
						\State overlap $\gets \False$
					\EndIf %End if component covered
				\EndIf %End if edge found that is in a component that is covered		
				\If{overlap}
					\State $R''_i \gets \overline{R''}$
					\State $\mathcal{R}'' \gets \mathcal{R}'' \cup R''_i$
					\State added $\gets \True$
					\State $\mathcal{C} \gets \overline{\mathcal{C}}$, $n \gets \overline{n}$
					\State $i \gets i+1$
				\ElsIf{$\Not$ overlap}
					\State $\overline{\mathcal{C}} \gets \mathcal{C}$, $\overline{n} \gets n$			
				\EndIf %End if overlap
			\EndIf %End if-elsif adj and diff cycles and one covered one not or none covered
			\State $k \gets k+1$
		\Until{$k \geq |\mathcal{L}|-1 \Or n = z$} %End do while loop
		\If{$n = z$}
			\State feasible, \textbf{go to} Line 93
		\ElsIf{added}
			\State remove edges that are in $\mathcal{R}''$ from $\mathcal{L}$
			\State added $\gets \False$
			\If{$|\mathcal{L}| < 2$}
				\State infeasible, not enough edges left to make another set, \textbf{end}
			\EndIf
		\ElsIf{$\Not$ added}
			\State infeasible, no new set created so no more sets exist, \textbf{end}
		\EndIf
	\EndWhile %End While !added
	\State \textit{Connecting the components together}
	\State let $v_j$ be the lower-indexed vertex of the $j$th edge in the set $R''_i$, and let $m(v_j)$ be the higher-indexed vertex of the $j$th edge in the set $R''_i$
	\For{$i \gets 1 \To |\mathcal{R}''|$}
		\For{$j \gets 1 \To |R''_i|$}
			\State $R' \gets R' \cup \{\{v_j, m(v_{j+1})\}\}$
		\EndFor
		\State $j \gets |R''_i|$
		\State $R' \gets R' \cup \{\{v_j, m(v_1)\}\}$
		\State remove all edges in $R''_i$ from $R'$
	\EndFor
	\Return either modified set $R'$ or statement of infeasibility
\end{algorithmic}
\label{alg:bcr}	
\end{algorithm}
\newpage

%--------------------------------------------------------------------------------------
\noindent Firstly, the edges in $R'$ are sorted into a list $\mathcal{L}$ where the lower-indexed vertices of the edges are in lexicographical order. Initially, the collection $\mathcal{R}''$ is empty, and variables are defined for the procedure. The variables $\mathcal{C}[j]$ $\forall j=1,\dotsc,z$ keeps track of the cyclic components that have been covered (i.e. which components of $G'$ have an edge that is in a subset in $\mathcal{R}''$). Here, $\mathcal{C}[j] =1$ if $\mathcal{R}''$ covers the component $C_j$ of $G'$, and $\mathcal{C}[j] =0$ otherwise. The variable $n$ is the number of components that are covered by $\mathcal{R}''$, i.e. $\sum_{j=1}^{z} \mathcal{C}[j]$.

The procedure begins by creating the first subset, $R''_1$, to be added to $\mathcal{R}''$. Firstly, BCR goes through the list of edges $\mathcal{L}$ to find two successive edges, $k$ and $k+1$, such that the lower-indexed vertex of the $k$th edge is adjacent to the higher-indexed vertex of the $k+1$th edge via an edge in $R$ (i.e. $\{v_k, m(v_{k+1})\} \in R$), and the two edges are in different components of $G'$ (Lines 7--8). If such edges are found, BCR adds these edges to the subset $R''_1$ (Lines 9--13). Then, BCR continues to add successive edges in $\mathcal{L}$ to $R''_1$ that meet the adjacency condition and are in different components of $G'$ (Lines 14--18). Once there are no more suitable edges to add to $R''_1$ that meet the conditions, $R''_1$ is added to the collection $\mathcal{R}''$ (Line 19). In the event that BCR has assessed all edges in $\mathcal{L}$ and has not produced a subset $R''_1$ (i.e. $\mathcal{R}'' = \emptyset$), then there is no way of connecting the components of $G'$ together; no solution exists and BCR terminates (Lines 20--21). However, if $\mathcal{R}''$ covers all $z$ components of $G'$ using just one subset $R''_1$, then no further subsets are needed and a feasible solution exists (Lines 22--23).

Otherwise, BCR must seek additional subsets to add to $\mathcal{R}''$. Any edges that appear in the subset $\mathcal{R}''_1 \subseteq \mathcal{R}''$ are removed from $\mathcal{L}$, and BCR starts the process again from the beginning of $\mathcal{L}$ (Line 28). As before, BCR seeks two successive edges in $\mathcal{L}$ that meet the adjacency condition and are in different components of $G'$ (Lines 32--33). If such edges are found, there are two possible scenarios.

The first is that one of the two edges is in a component of $G'$ that is already covered by $\mathcal{R}''$, whilst the other edge is in a component not yet covered by $\mathcal{R}''$ (Line 34). In this case, the two edges are added to a new subset, $R''_i$, and BCR adds successive edges in $\mathcal{L}$ to $R''_i$ provided the adjacency condition holds and the edges are in components of $G'$ not covered by $\mathcal{R}''$. (Lines 35--44). Then, $R''_i$ is added to $\mathcal{R}''$ (Line 45). 

The second scenario arises when neither of the two edges are in a component covered by $\mathcal{R}''$ (Line 49). In this case, BCR makes use of a temporary subset $\overline{R''}$ with the aim of finding an edge that is in a component covered by $\mathcal{R}''$ (an ``overlap''). So, the two edges are added to the temporary subset $\overline{R''}$ (Line 51) and, as before, successive edges in $\mathcal{L}$ that meet the adjacency condition and are in components of $G'$ not yet covered by $\mathcal{R}''$ are added to $\overline{R''}$ (Lines 55--59). If, however, BCR comes upon an edge that is in a component that has been temporarily marked as covered (i.e. $\overline{\mathcal{C}}[r] = 1$), then BCR checks whether the edge is in a component that has actually been covered by $\mathcal{R}''$ previously (i.e. whether $\mathcal{C}[r] = 1$). If true, then an overlapping edge has been found, and the edge is added to $\overline{R''}$ (Lines 60--64). Then, BCR proceeds to add suitable successive edges in $\mathcal{L}$ to $\overline{R''}$ that are in components not covered by $\mathcal{R}''$ (Lines 65--71). This temporary subset then becomes a subset $R''_i$ which is added to $\mathcal{R}''$ (Lines 74--79). Otherwise, if the edge is not in a component that is covered by $\mathcal{R}''$ (i.e. $\mathcal{C}[r] = 0$), then the temporary subset $\overline{R''}$ does not meet the conditions for a feasible subset, and BCR ceases to add any further edges to $\overline{R''}$ (Lines 72--73). 

This process of searching through $\mathcal{L}$ and creating edge subsets continues until either (a) the penultimate edge is reached, or (b) the collection $\mathcal{R}''$ covers all $z$ components of $G'$ (Line 83). If the latter, then a feasible solution exists (Lines 84--85). On the other hand, if BCR has added a subset to $\mathcal{R}''$ then the edges in $\mathcal{R}''$ are removed from $\mathcal{L}$. If less than two edges remain in $\mathcal{L}$, then there are not enough edges left in $\mathcal{L}$ to form another edge subset, and as $\mathcal{R}''$ does not cover all $z$ components of $G'$ no feasible solution exists, and so BCR terminates (Lines 89--90). Otherwise, BCR resumes the process of finding new subsets starting from the beginning of $\mathcal{L}$ (Line 28). If no new subset has been added to $\mathcal{R}''$, then no more suitable subsets can be formed from the edges in $\mathcal{L}$, and so no feasible solution exists (Lines 91--92). 

From a feasible collection $\mathcal{R}''$ covering all $z$ components of $G'$, BCR uses each subset $R''_i \subseteq \mathcal{R}''$ to modify the subset $R'$ (Line 93). For each edge in $R''_i$ in turn, the edge from $R \backslash R'$ connecting the lower-indexed vertex of the edge to the higher-indexed vertex of the next edge is added to $R'$ (Lines 95--99). The edges in the subset $R''_i$ are then removed from $R'$ (Line 100). Once all subsets in $\mathcal{R}''$ have been operated on in this manner, the resulting modified matching $R'$ forms an alternating Hamiltonian cycle in $G$ with the edges in $B$. This process is illustrated in Fig. 6 of the article.








	
\end{document}	