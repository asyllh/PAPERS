%---- PREAMBLE ----%

\usepackage[english]{babel}     
\usepackage[T1]{fontenc}        
\usepackage[utf8]{inputenc}  

% Allows separate files to be used in the main file
\usepackage{subfiles}

% Algorithms
\usepackage{algorithm}
\usepackage[noend]{algpseudocode}
\renewcommand{\algorithmicrequire}{\textbf{Input:}}
\renewcommand{\algorithmicensure}{\textbf{Output:}}
\newcommand{\algorithmicbreak}{\State \textbf{break}}
\newcommand{\Break}{\algorithmicbreak}
\renewcommand{\algorithmicreturn}{\State \textbf{return}}
\newcommand{\algorithmicto}{\textbf{ to }}
\newcommand{\To}{\algorithmicto}
\newcommand{\algorithmicrun}{\State \textbf{call }}
\newcommand{\Run}{\algorithmicrun}
\newcommand{\algorithmicoutput}{\textbf{output }}
\newcommand{\Output}{\algorithmicoutput}
\newcommand{\algorithmictrue}{\textbf{true}}
\newcommand{\True}{\algorithmictrue}
\newcommand{\algorithmicfalse}{\textbf{false}}
\newcommand{\False}{\algorithmicfalse}
\newcommand{\algorithmicand}{\textbf{ and }}
\newcommand{\AAnd}{\algorithmicand}
\newcommand{\algorithmicnot}{\textbf{not}}
\newcommand{\Not}{\algorithmicnot}
\newcommand{\algorithmicor}{\textbf{ or }}
\newcommand{\Or}{\algorithmicor}

% Contains advanced math extensions
\usepackage{amsmath}

% Introduces the *proof* environment and the \theoremstyle command
\usepackage{amsthm}

% Adds new symbols to be used in math mode, e.g. \mathbb
\usepackage{amssymb}

% To declare multiple authors
%\usepackage{authblk}

% Provides extra comands as well as optimisation for producing tables
\usepackage{booktabs}
\newcommand{\ra}[1]{\renewcommand{\arraystretch}{#1}}

% Allows customisation of appearance and placements for figures/tables etc.
\usepackage{caption}

\usepackage{comment}

% Adds support for arbitrarily-deep nested lists
\usepackage[inline]{enumitem}

%Support for changing size of font in table footnotes
%\usepackage{etoolbox}

% Improves the interface for defining floating objects such as figures/tables
\usepackage{float}

%\usepackage{fullpage}

% For easy management of document margins and the document page size
\usepackage[top=25mm, bottom=35mm, left=30mm, right=30mm]{geometry}

% Allows insertion of graphic files within a document
\usepackage{graphicx}

% Manage links within the document or to any URL when you compile in PDF
\usepackage[colorlinks]{hyperref} 
\usepackage[dvipsnames]{xcolor}

% Tikz colours, used in tikz figures only
\colorlet{tRed}{Red} %tikz color
\definecolor{tOrange}{HTML}{FA7602} %tikz color
\definecolor{tOrange1}{HTML}{FAA51C} %tikz color %FAA51C
\definecolor{tLightOrange}{HTML}{FFE3B2}
\definecolor{tYellow}{HTML}{FFFF7F}
\definecolor{tLightYellow}{HTML}{FFFBBD} %tikz color
\colorlet{tGreen}{LimeGreen!10!OliveGreen!70!White}
\colorlet{tGreen1}{LimeGreen!20!OliveGreen!50!White}
\definecolor{tGreen2}{HTML}{569909} %tikz color
\definecolor{tLightGreen}{HTML}{D3ECAA}
\definecolor{tDarkGreen}{HTML}{9BBF86} %tikz color %9BBF86
\definecolor{tTurquoise}{HTML}{ACFFEF} %tikz color %6FC4DD %7FE8F3
\colorlet{tBlue}{RoyalBlue!35!Cerulean} %tikz color
\definecolor{tLightBlue}{HTML}{CEF0FF} %tikz color %CEF0FF
\definecolor{tDarkBlue}{HTML}{95B5FF} %tikz color %95B5FF
\definecolor{tPurple}{HTML}{7356AF} %tikz color %ECDBFF
\definecolor{tPurple1}{HTML}{A288F5} %tikz color %9F85F5
\definecolor{tLightPurple}{HTML}{D8CFFD} %tikz color %ECDBFF
\definecolor{tPink}{HTML}{DF4979} %tikz color %F98F96
\colorlet{tPink1}{OrangeRed!50!WildStrawberry!60!White}
\definecolor{tLightPink}{HTML}{FFD4EB} %tikz color
\definecolor{tPeach}{HTML}{F98F96} %tikz color %F98F96



% Text colours
\colorlet{myRed}{Red!60!OrangeRed} 
\definecolor{myDarkOrange}{HTML}{FF5500}
\definecolor{myOrange}{HTML}{EE7001} %FA7602
\definecolor{myYellow}{HTML}{F0CF11} %E6C612
\definecolor{myGreen}{HTML}{569909} %569909
\definecolor{myDarkGreen}{HTML}{008000}
\definecolor{myAqua}{HTML}{00B1BA} %02BEB8 %00b1ba
\definecolor{myBlue}{HTML}{0F88F1} %00B3FF %0095ff
\definecolor{myDarkBlue}{HTML}{546EFF} 
\colorlet{myPurple}{Orchid} 
\definecolor{myPink}{HTML}{FF65C2} %FF65C2 %FF53AC
\definecolor{myMagenta}{HTML}{FF427E} %FF53AC %%FF557F
\colorlet{myGray}{Gray!90!White}

\hypersetup{
	linkcolor=myPurple,
	citecolor=myAqua,
	urlcolor=blue
}
\newcommand{\intro}[1]{{\color{myDarkBlue}#1}} % \intro{<text>}, makes <text> dark blue, use to highlight things to do, urgent, or revisit in Introduction section

\newcommand{\lit}[1]{{\color{myOrange}#1}} % \lit{<text>}, makes <text> orange, use to highlight things to do, urgent, or revisit in Lit Review section

\newcommand{\ahc}[1]{{\color{myBlue}#1}} % \ahc{<text>}, makes <text> blue, use to highlight things to do, urgent, or revisit in AHC section

\newcommand{\heur}[1]{{\color{myPink}#1}} % \heur{<text>}, makes <text> pink, use to highlight things to do, urgent, or revisit in Heuristics section

\newcommand{\ea}[1]{{\color{myRed}#1}} % \ea{<text>}, makes <text> red, use to highlight things to do, urgent, or revisit in EA section

\newcommand{\cmsa}[1]{{\color{myGreen}#1}} % \hyip{<text>}, makes <text> green, use to highlight things to do, urgent, or revisit in Hybrid/IP section

\newcommand{\msc}[1]{{\color{myPurple}#1}} % \msc{<text>}, makes <text> purple, use to highlight things to do, urgent, or revisit in Modified SCPP section

\newcommand{\conc}[1]{{\color{myDarkBlue}#1}} % \conc{<text>}, makes <text> dark blue, use to highlight things to do, urgent, or revisit in Conclusion section

\newcommand{\note}[1]{{\color{myMagenta}#1}} % \note{<text>}, makes <text> magenta, use to highlight things to do, urgent, or revisit in any section

% Successor of amsmath
\usepackage{mathtools}

\usepackage{multirow}

% No indentation, space between paragraphs
%\usepackage{parskip}

\usepackage[authoryear,round]{natbib}

\usepackage{setspace}
\onehalfspacing


%Include standalone .tex files
\usepackage{standalone}

% Define multiple floats (figures/tables) within one environment with individual captions 1a, 1b etc
\usepackage{subcaption}

%\usepackage[caption=false,font=footnotesize]{subfig}

\usepackage{tabularx}

\usepackage{threeparttable}
%\appto\TPTnoteSettings{\footnotesize} % Change font size of table footnotes to footnotesize

\usepackage{tikz}
\usetikzlibrary{shapes.geometric}
\usetikzlibrary{arrows,decorations.markings}
\usetikzlibrary{patterns}
\usetikzlibrary{decorations.pathreplacing,decorations.markings}
\usepgflibrary{shapes.geometric}
\usepackage{rotating}

\usepackage{wrapfig}

% Need \usepackage{mathtools} to use floor/ceil below.
% Example: \floor*{\frac{x}{2}}, \ceil*{\frac{x}{2}}
% The asterisk resizes floor/ceil brackets.
\DeclarePairedDelimiter{\floor}{\lfloor}{\rfloor}
\DeclarePairedDelimiter{\ceil}{\lceil}{\rceil}

%Theorem style
%\theoremstyle{plain}% default
\newtheorem{theorem}{Theorem}
%\newtheorem{corollary}{Corollary}
\newtheorem{definition}{Definition}

%\theoremstyle{definition}
%\newtheorem{definition}{Definition}
%\newtheorem{proposition}{Proposition}
%\newtheorem{exmp}{Example}[section]